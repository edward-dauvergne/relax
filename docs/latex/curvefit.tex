% Relaxation curve-fitting.
%%%%%%%%%%%%%%%%%%%%%%%%%%%

\chapter[Relaxation curve-fitting]{The $\Rone$ and $\Rtwo$ relaxation rates -- relaxation curve-fitting}
\index{relaxation curve-fitting|textbf}



% Introduction.
%%%%%%%%%%%%%%%

\section{Introduction}

Relaxation curve-fitting involves a number of steps including the loading of data, the calculation of both the average peak intensity\index{peak!intensity} across replicated spectra and the standard deviations\index{standard deviation} of those peak intensities, selection of the experiment type, optimisation of the parameters of the fit, Monte Carlo simulations\index{Monte Carlo simulation} to find the parameter errors, and saving and viewing the results.  To simplify the process a sample script will be followed step by step as was done with the NOE calculation.



% The sample script.
%%%%%%%%%%%%%%%%%%%%

\section{The sample script}

\begin{exampleenv}
\# Script for relaxation curve-fitting. \\
 \\
\# Create the `rx' data pipe. \\
pipe.create(`rx', `relax\_fit') \\
 \\
\# Load the backbone amide 15N spins from a PDB file. \\
structure.read\_pdb(`Ap4Aase\_new\_3.pdb') \\
structure.load\_spins(spin\_id=`@N') \\
 \\
\# Spectrum names. \\
names = [ \\
\hspace*{4ex} `T2\_ncyc1\_ave', \\
\hspace*{4ex} `T2\_ncyc1b\_ave', \\
\hspace*{4ex} `T2\_ncyc2\_ave', \\
\hspace*{4ex} `T2\_ncyc4\_ave', \\
\hspace*{4ex} `T2\_ncyc4b\_ave', \\
\hspace*{4ex} `T2\_ncyc6\_ave', \\
\hspace*{4ex} `T2\_ncyc9\_ave', \\
\hspace*{4ex} `T2\_ncyc9b\_ave', \\
\hspace*{4ex} `T2\_ncyc11\_ave', \\
\hspace*{4ex} `T2\_ncyc11b\_ave' \\
] \\
 \\
\# Relaxation times (in seconds). \\
times = [ \\
\hspace*{4ex} 0.0176, \\
\hspace*{4ex} 0.0176, \\
\hspace*{4ex} 0.0352, \\
\hspace*{4ex} 0.0704, \\
\hspace*{4ex} 0.0704, \\
\hspace*{4ex} 0.1056, \\
\hspace*{4ex} 0.1584, \\
\hspace*{4ex} 0.1584, \\
\hspace*{4ex} 0.1936, \\
\hspace*{4ex} 0.1936 \\
] \\
 \\
\# Loop over the spectra. \\
for i in xrange(len(names)): \\
\hspace*{4ex} \# Load the peak intensities. \\
\hspace*{4ex} spectrum.read\_intensities(file=names[i]+`.list', dir=data\_path, spectrum\_id=names[i], int\_method=`height') \\
 \\
\hspace*{4ex} \# Set the relaxation times. \\
\hspace*{4ex} relax\_fit.relax\_time(time=times[i], spectrum\_id=names[i]) \\
 \\
\# Specify the duplicated spectra. \\
spectrum.replicated(spectrum\_ids=[`T2\_ncyc1\_ave', `T2\_ncyc1b\_ave']) \\
spectrum.replicated(spectrum\_ids=[`T2\_ncyc4\_ave', `T2\_ncyc4b\_ave']) \\
spectrum.replicated(spectrum\_ids=[`T2\_ncyc9\_ave', `T2\_ncyc9b\_ave']) \\
spectrum.replicated(spectrum\_ids=[`T2\_ncyc11\_ave', `T2\_ncyc11b\_ave']) \\
 \\
\# Peak intensity error analysis. \\
spectrum.error\_analysis() \\
 \\
\# Deselect unresolved spins. \\
deselect.read(file=`unresolved', mol\_name\_col=1, res\_num\_col=2, res\_name\_col=3, spin\_num\_col=4, spin\_name\_col=5) \\
 \\
\# Set the relaxation curve type. \\
relax\_fit.select\_model(`exp') \\
 \\
\# Grid search. \\
grid\_search(inc=11) \\
 \\
\# Minimise. \\
minimise(`simplex', scaling=False, constraints=False) \\
 \\
\# Monte Carlo simulations. \\
monte\_carlo.setup(number=500) \\
monte\_carlo.create\_data() \\
monte\_carlo.initial\_values() \\
minimise(`simplex', scaling=False, constraints=False) \\
monte\_carlo.error\_analysis() \\
 \\
\# Save the relaxation rates. \\
value.write(param=`rx', file=`rx.out', force=True) \\
 \\
\# Save the results. \\
results.write(file=`results', force=True) \\
 \\
\# Create Grace plots of the data. \\
grace.write(y\_data\_type=`chi2', file=`chi2.agr', force=True)    \# Minimised chi-squared value. \\
grace.write(y\_data\_type=`i0', file=`i0.agr', force=True)    \# Initial peak intensity. \\
grace.write(y\_data\_type=`rx', file=`rx.agr', force=True)    \# Relaxation rate. \\
grace.write(x\_data\_type=`relax\_times', y\_data\_type=`intensities', file=`intensities.agr', force=True)    \# Average peak intensities. \\
grace.write(x\_data\_type=`relax\_times', y\_data\_type=`intensities', norm=True, file=`intensities\_norm.agr', force=True)    \# Average peak intensities (normalised). \\
 \\
\# Display the Grace plots. \\
grace.view(file=`chi2.agr') \\
grace.view(file=`i0.agr') \\
grace.view(file=`rx.agr') \\
grace.view(file=`intensities.agr') \\
grace.view(file=`intensities\_norm.agr') \\
 \\
\# Save the program state.
state.save(`rx.save', force=True)
\end{exampleenv}



% Initialisation of the data pipe and loading of the data.
%%%%%%%%%%%%%%%%%%%%%%%%%%%%%%%%%%%%%%%%%%%%%%%%%%%%%%%%%%

\section{Initialisation of the data pipe and loading of the data}

The start of this sample script is very similar to that of the NOE calculation on page~\pageref{NOE initialisation}.  The command

\begin{exampleenv}
pipe.create(`rx', `relax\_fit')
\end{exampleenv}

initialises the data pipe labelled \texttt{`rx'}.  The data pipe type is set to relaxation curve-fitting by the argument \texttt{`relax\_fit'}.  The backbone amide nitrogen sequence is extracted from a PDB\index{PDB} file using the same commands as the NOE calculation script

\example{structure.read\_pdb(name, `Ap4Aase\_new\_3.pdb')}
\index{PDB}

\example{structure.load\_spins(spin\_id=`@N')}

To load the peak intensities\index{peak!intensity} into relax the user function \texttt{spectrum.read\_intensities} is executed.  Important keyword arguments to this command are the file name and directory, the spectrum identification string and the relaxation time period of the experiment in seconds.  By default the file format will be automatically detected.  Currently Sparky\index{software!Sparky}, XEasy\index{software!XEasy}, NMRView\index{software!NMRView}, and generic columnar formatted peak lists are supported.  To be able to import any other type of format please send an email to the relax development mailing list\index{mailing list!relax-devel} with the details of the format.  Adding support for new formats is trivial.  The following series of commands will load peak intensities from six different relaxation periods, four of which have been duplicated, from Sparky peak lists with the peak heights in the 10$^\textrm{th}$ column.

\begin{exampleenv}
spectrum.read\_intensities(file=`T2\_ncyc1.list',   spectrum\_id=`1',   relax\_time=0.0176, int\_col=10) \\
spectrum.read\_intensities(file=`T2\_ncyc1b.list',  spectrum\_id=`1b',  relax\_time=0.0176, int\_col=10) \\
spectrum.read\_intensities(file=`T2\_ncyc2.list',   spectrum\_id=`2',   relax\_time=0.0352, int\_col=10) \\
spectrum.read\_intensities(file=`T2\_ncyc4.list',   spectrum\_id=`4',   relax\_time=0.0704, int\_col=10) \\
spectrum.read\_intensities(file=`T2\_ncyc4b.list',  spectrum\_id=`4b',  relax\_time=0.0704, int\_col=10) \\
spectrum.read\_intensities(file=`T2\_ncyc6.list',   spectrum\_id=`6',   relax\_time=0.1056, int\_col=10) \\
spectrum.read\_intensities(file=`T2\_ncyc9.list',   spectrum\_id=`9',   relax\_time=0.1584, int\_col=10) \\
spectrum.read\_intensities(file=`T2\_ncyc9b.list',  spectrum\_id=`9b',  relax\_time=0.1584, int\_col=10) \\
spectrum.read\_intensities(file=`T2\_ncyc11.list',  spectrum\_id=`11',  relax\_time=0.1936, int\_col=10) \\
spectrum.read\_intensities(file=`T2\_ncyc11b.list', spectrum\_id=`11b', relax\_time=0.1936, int\_col=10) \\
 \\
spectrum.replicated(spectrum\_ids=[`T2\_ncyc1\_ave', `T2\_ncyc1b\_ave']) \\
spectrum.replicated(spectrum\_ids=[`T2\_ncyc4\_ave', `T2\_ncyc4b\_ave']) \\
spectrum.replicated(spectrum\_ids=[`T2\_ncyc9\_ave', `T2\_ncyc9b\_ave']) \\
spectrum.replicated(spectrum\_ids=[`T2\_ncyc11\_ave', `T2\_ncyc11b\_ave'])
\end{exampleenv}

For the Sparky peak lists, by default relax assumes that the intensity value is in the 4$^\textrm{th}$ column.  A typical file looks like:

{\scriptsize \begin{verbatim}
     Assignment         w1         w2   Data Height

        LEU3N-HN    122.454      8.397       129722
        GLY4N-HN    111.999      8.719       422375
        SER5N-HN    115.085      8.176       384180
        MET6N-HN    120.934      8.812       272100
        ASP7N-HN    122.394      8.750       174970
        SER8N-HN    113.916      7.836       218762
       GLU11N-HN    122.194      8.604        30412
       GLY12N-HN    110.525      9.028        90144
\end{verbatim}}

By supplying the \texttt{`int\_col'} argument to the \texttt{spectrum.read\_intensities} user function, this can be changed.  A typical XEasy file will look like:

{\scriptsize \begin{verbatim}
 No.  Color    w1      w2     ass. in w1     ass. in w2    Volume     Vol. Err.  Method  Comment

   2    2    10.014 134.221   HN  21 LEU      N  21 LEU    7.919e+03  0.00e+00     m
   3    2    10.481 132.592  HE1  79 TRP    NE1  79 TRP    1.532e+04  0.00e+00     m
  17    2     9.882 129.041   HN 110 PHE      N 110 PHE    9.962e+03  0.00e+00     m
  18    2     8.757 128.278   HN  52 ASP      N  52 ASP    2.041e+04  0.00e+00     m
  19    2    10.086 128.297   HN  69 SER      N  69 SER    9.305e+03  0.00e+00     m
  20    3     9.111 127.707   HN  15 ARG      N  15 ARG    9.714e+03  0.00e+00     m
\end{verbatim}}

where the peak height is in the `Volume' column.  And for an NMRView file:

{\scriptsize \begin{verbatim}
label dataset sw sf
H1 N15
cNTnC_noe0.nv
2505.63354492 1369.33557129
499.875 50.658000946
H1.L H1.P H1.W H1.B H1.E H1.J H1.U N15.L N15.P N15.W N15.B N15.E N15.J N15.U vol int stat comment flag0
0 {70.HN} 10.75274 0.02954 0.05379 ++ 0.0 {} {70.N} 116.37241 0.23155 0.35387 ++ 0.0 {} -6.88333129883 -0.1694 0 {} 0
1 {72.HN} 9.67752 0.03308 0.05448 ++ 0.0 {} {72.N} 126.41302 0.27417 0.37217 ++ 0.0 {} -5.49038267136 -0.1142 0 {} 0
2 {} 8.4532 0.02331 0.05439 ++ 0.0 {} {} 122.20137 0.38205 0.33221 ++ 0.0 {} -2.58034267191 -0.1320 0 {} 0
\end{verbatim}}


% The rest of the setup.
%%%%%%%%%%%%%%%%%%%%%%%%

\section{The rest of the setup}

Once all the peak intensity data has been loaded a few calculations are required prior to optimisation.  Firstly the peak intensities for individual spins needs to be averaged across replicated spectra.  The peak intensity errors also have to be calculated using the standard deviation formula.  These two operations are executed by the user function

\example{spectrum.error\_analysis()}

Any spins which cannot be resolved due to peak overlap were included in a file called \texttt{`unresolved'}.  This file can consist of optional columns of the molecule name, the residue name and number, and the spin name and number.  The matching spins are excluded from the analysis by the user function

\example{deselect.read(file=`unresolved', mol\_name\_col=1, res\_num\_col=2, res\_name\_col=3, spin\_num\_col=4, spin\_name\_col=5)}

Finally the experiment type is specified by the command

\example{relax\_fit.select\_model(`exp')}

The argument \texttt{`exp'} sets the relaxation curve to a two parameter \{$\mathrm{R}_x$, $I_0$\} exponential which decays to zero.  The formula of this function is
\begin{equation}
 I(t) = I_0 e^{-\mathrm{R}_x \cdot t},
\end{equation}

\noindent where $I(t)$ is the peak intensity at any time point $t$, $I_0$ is the initial intensity, and $\mathrm{R}_x$ is the relaxation rate (either the $\Rone$ or $\Rtwo$).  Changing the user function argument to \texttt{`inv'} will select the inversion recovery experiment.  This curve consists of three paremeters \{$\Rone$, $I_0$, $I_{\infty}$\} and does not decay to zero.  The formula is
\begin{equation}
 I(t) = I_{\infty} - I_0 e^{-\Rone \cdot t}.
\end{equation}



% Optimisation.
%%%%%%%%%%%%%%%

\section{Optimisation}

Now that everything has been setup minimision can be used to optimise the parameter values.  Firstly a grid search is applied to find a rough starting position for the subsequent optimisation algorithm.  Eleven increments per dimension of the model (in this case the two dimensions \{$\mathrm{R}_x$, $I_0$\}) is sufficient.  The user function for executing the grid search is

\example{grid\_search(inc=11)}

The next step is to select one of the minimisation algorithms to optimise the model parameters.  Currently for relaxation curve-fitting only simplex minimisation is supported.  This is because the relaxation curve-fitting C module is incomplete only implementing the chi-squared function.  The chi-squared gradient (the vector of first partial derivatives) and chi-squared Hessian (the matrix of second partial derivatives) are not yet implemented in the C modules and hence optimisation algorithms which only employ function calls are supported.  Simplex minimisation is the only technique in relax which fits this criteron.  In addition constraints cannot be used as the constraint algorithm is dependent on gradient calls.  Therefore the minimisation command for relaxation curve-fitting is forced to be

\example{minimise(`simplex', constraints=False)}



% Error analysis.
%%%%%%%%%%%%%%%%%

\section{Error analysis}

Only one technique adequately estimates parameter errors when the parameter values where found by optimisation -- Monte Carlo simulations\index{Monte Carlo simulation}.  In relax this can be implemented by using a series of functions from the \texttt{monte\_carlo} user function class.  Firstly the number of simulations needs to be set

\example{monte\_carlo.setup(number=500)}

For each simulation, randomised relaxation curves will be fit using exactly the same methodology as the original exponential curves.  These randomised curves are created by back calculation from the fitted model parameter values and then each point on the curve randomised using the error values set earlier in the script

\example{monte\_carlo.create\_data()}

As a grid search for each simulation would be too computationally expensive, the starting point for optimisation for each simulation can be set to the position of the optimised parameter values of the model

\example{monte\_carlo.initial\_values()}

Then exactly the same optimisation as was used for the model can be performed

\example{minimise(`simplex', constraints=False)}

The parameter errors are then determined as the standard deviation of the optimised parameter values of the simulations

\example{monte\_carlo.error\_analysis()}


% R1 analysis screenshot
\begin{figure}
\centerline{\includegraphics[width=\textwidth, bb=14 14 1065 768]{graphics/screenshots/analysis_r1.eps.gz}}
\caption[GUI screenshot -- $\Rone$ analysis]{Screenshot of the relax GUI interface -- the $\Rone$ analysis.}\label{fig: screenshot: R1 analysis}
\end{figure}


% Finishing off.
%%%%%%%%%%%%%%%%

\section{Finishing off}

To finish off, the script first saves the relaxation rates together with their errors in a simple text file

\example{value.write(param=`rx', file=`rx.out', force=True)}

Grace plots are created and viewed

\example{grace.write(y\_data\_type=`chi2', file=`chi2.agr', force=True)}

\example{grace.write(y\_data\_type=`i0', file=`i0.agr', force=True)}

\example{grace.write(y\_data\_type=`rx', file=`rx.agr', force=True)}

\example{grace.write(x\_data\_type=`relax\_times', y\_data\_type=`intensities', file=`intensities.agr', force=True)}

\example{grace.write(x\_data\_type=`relax\_times', y\_data\_type=`intensities', norm=True, file=`intensities\_norm.agr', force=True)}

\example{grace.view(file=`chi2.agr')}

\example{grace.view(file=`i0.agr')}

\example{grace.view(file=`rx.agr')}

\example{grace.view(file=`intensities.agr')}

\example{grace.view(file=`intensities\_norm.agr')}

and finally the program state is saved for future reference

\example{state.save(file=`rx.save', force=True)}


% R2 analysis screenshot
\begin{figure}
\centerline{\includegraphics[width=\textwidth, bb=14 14 1065 768]{graphics/screenshots/analysis_r2.eps.gz}}
\caption[GUI screenshot -- $\Rtwo$ analysis]{Screenshot of the relax GUI interface -- the $\Rtwo$ analysis.}\label{fig: screenshot: R2 analysis}
\end{figure}



% GUI.
%%%%%%

\section{The GUI auto-analysis}

The $\Rone$ and $\Rtwo$ relaxation rates can be calculated using the relax GUI (see Figures~\ref{fig: screenshot: R1 analysis} and~\ref{fig: screenshot: R2 analysis}).  These auto-analyses can be selected using the analysis selection wizard (Figure~\ref{fig: screenshot: analysis wizard} on page~\pageref{fig: screenshot: analysis wizard}).  Just as with the steady-state NOE, these auto-analyses are very similar in spirit to the sample script described in this chapter, though the Grace 2D visualisation is more advanced.  If you have read this chapter, the usage of these analyses should be self explanatory.


% Final checks.
%%%%%%%%%%%%%%%

\section{Final checks}

To be sure that the data has been properly collected and that no instrumentation or pulse sequence timing errors have occurred, it is essential to carefully check the \texttt{intensities.agr} and \texttt{intensities\_norm.agr} 2D Grace files.  These are plots of the decay curves for each spin system analysed, and any non-exponential behaviour should be clearly visible.

Note that errors resulting in systematic bias in the data -- for example if temperature control (single-scan interleaving or temperature compensation blocks) or per-experiment/per-spectrometer temperature calibration on MeOH or ethylene glycol have not been performed -- will not be detected by looking at the decay curves.  See the \texttt{relax\_data.temp\_calibration} and \texttt{relax\_data.temp\_control} user function documentation for more details.
