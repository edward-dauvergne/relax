% Model-free analysis.
%%%%%%%%%%%%%%%%%%%%%%

\chapter{Model-free analysis}
\label{ch: model-free}
\index{model-free analysis|textbf}


\begin{figure*}[h]
\includegraphics[width=5cm, bb=0 0 1701 1701]{graphics/analyses/model_free/model_free_600x600.eps.gz}
\end{figure*}


% Theory.
%%%%%%%%%

\section{Theory}



% The chi-squared function.
\begin{latexonly}
    \subsection{The chi-squared function -- $\chi^2(\theta)$}
\end{latexonly}
\begin{htmlonly}
    \subsection{The chi-squared function -- $chi^2(theta)$}
\begin{htmlonly}

\index{chi-squared|textbf}

For the minimisation\index{minimisation} of the model-free models a chain of calculations, each based on a different theory, is required.  At the highest level the equation which is actually minimised is the chi-squared function
\begin{equation} \label{eq: chi-squared}
 \chi^2(\theta) = \sum_{i=1}^n \frac{(\Ri - \Ri(\theta))^2}{\sigma_i^2},
\end{equation}

\noindent where the index $i$ is the summation index ranging over all the experimentally collected relaxation data of all spins used in the analysis; $\Ri$ belongs to the relaxation data set R for an individual spin, a collection of spins, or the entire macromolecule and includes the $\Rone$, $\Rtwo$, and NOE data at all field strengths; $\Ri(\theta)$ is the back-calculated relaxation value belonging to the set R$(\theta)$; $\theta$ is the model parameter vector which when minimised is denoted by $\hat\theta$; and $\sigma_i$ is the experimental error.

The significance of the chi-squared equation~\eqref{eq: chi-squared} is that the function returns a single value which is then minimised by the optimisation algorithm to find the model-free parameter values of the given model.



% The transformed relaxation equations.
\begin{latexonly}
    \subsection{The transformed relaxation equations -- $\Ri(\theta)$}
\end{latexonly}
\begin{htmlonly}
    \subsection{The transformed relaxation equations -- R$_i(theta)$}
\end{htmlonly}

The chi-squared equation is itself dependent on the relaxation equations through the back-calculated relaxation data R$(\theta)$.  Letting the relaxation values of the set R$(\theta)$ be the $\Rone(\theta)$, $\Rtwo(\theta)$, and NOE$(\theta)$ an additional layer of abstraction can be used to simplify the calculation of the gradients and Hessians.  This involves decomposing the NOE equation into the cross relaxation rate constant $\crossrate(\theta)$ and the auto relaxation rate $\Rone(\theta)$.  Taking equation~\eqref{eq: NOE} below the transformed relaxation equations are
\begin{subequations}
\begin{align}
    \Rone(\theta) &= \Rone'(\theta), \\
    \Rtwo(\theta) &= \Rtwo'(\theta), \\
    \mathrm{NOE}(\theta)  &= 1 + \frac{\gH}{\gX} \frac{\crossrate(\theta)}{\Rone(\theta)}.
\end{align}
\end{subequations}

\noindent whereas the relaxation equations are the $\Rone(\theta)$, $\Rtwo(\theta)$, $\crossrate(\theta)$.



% The relaxation equations.
\begin{latexonly}
    \subsection{The relaxation equations -- $\Ri'(\theta)$}
\end{latexonly}
\begin{htmlonly}
    \subsection{The relaxation equations -- R$_i$ $prime(theta)$}
\end{htmlonly}

The relaxation values of the set R$'(\theta)$ include the spin-lattice\index{relaxation rate!spin-lattice}, spin-spin\index{relaxation rate!spin-spin}, and cross-relaxation rates\index{relaxation rate!cross rate} at all field strengths.  These rates are respectively \citep{Abragam61}
\begin{subequations}
\begin{align}
    \Rone(\theta) &= d \Big( J(\omega_H - \omega_X) + 3J(\omega_X) + 6J(\omega_H + \omega_X) \Big) + cJ(\omega_X),     \label{eq: R1} \\
    \Rtwo(\theta) &= \frac{d}{2} \Big( 4J(0) + J(\omega_H - \omega_X) + 3J(\omega_X) + 6J(\omega_H)                    \nonumber \\
        & \quad + 6J(\omega_H + \omega_X) \Big) + \frac{c}{6} \Big( 4J(0) + 3J(\omega_X) \Big) + R_{ex},              \label{eq: R2} \\  
    \crossrate(\theta) &= d \Big( 6J(\omega_H + \omega_X) - J(\omega_H - \omega_X) \Big),                              \label{eq: sigma_NOE}
\end{align}
\end{subequations}

\noindent where $J(\omega)$ is the power spectral density function and $R_{ex}$ is the relaxation due to chemical exchange.  The dipolar and CSA constants are defined in SI units as
\begin{gather}
 d = \frac{1}{4} \left(\frac{\mu_0}{4\pi}\right)^2 \frac{(\gH \gX \hbar)^2}{\langle r^6 \rangle}, \label{eq: dipolar constant} \\
 c = \frac{(\omega_H \Delta\sigma)^2}{3}, \label{eq: CSA constant}
\end{gather}

\noindent where $\mu_0$ is the permeability of free space, $\gH$ and $\gX$ are the gyromagnetic ratios of the $H$ and $X$ spins respectively, $\hbar$ is Plank's constant divided by $2\pi$, $r$ is the bond length, and $\Delta\sigma$ is the chemical shift anisotropy measured in ppm.  The cross-relaxation rate $\crossrate$\index{relaxation rate!cross-relaxation|textbf} is related to the steady state NOE by the equation
\begin{equation} \label{eq: NOE}
 \mathrm{NOE}(\theta) = 1 + \frac{\gH}{\gX} \frac{\crossrate(\theta)}{\Rone(\theta)}.
\end{equation}



% The spectral density functions.
\begin{latexonly}
    \subsection{The spectral density functions -- $J(\omega)$}
\end{latexonly}
\begin{htmlonly}
    \subsection{The spectral density functions -- $J(omega)$}
\end{htmlonly}

The relaxation equations are themselves dependent on the calculation of the spectral density values $J(\omega)$.  Within model-free analysis these are modelled by the original model-free formula \citep{LipariSzabo82a, LipariSzabo82b}
\begin{equation} \label{eq: J(w) model-free generic}
    J(\omega) = \frac{2}{5} \sum_{i=-k}^k c_i \cdot \tau_i \Bigg(
        \frac{S^2}{1 + (\omega \tau_i)^2}
        + \frac{(1 - S^2)(\tau_e + \tau_i)\tau_e}{(\tau_e + \tau_i)^2 + (\omega \tau_e \tau_i)^2}
    \Bigg),
\end{equation}

\noindent where $S^2$ is the square of the Lipari and Szabo generalised order parameter and $\tau_e$ is the effective correlation time.  The order parameter reflects the amplitude of the motion and the correlation time in an indication of the time scale of that motion.  The theory was extended by \citet{Clore90a} by the modelling of two independent internal motions using the equation
\begin{multline} \label{eq: J(w) model-free ext generic}
    J(\omega) = \frac{2}{5} \sum_{i=-k}^k c_i \cdot \tau_i \Bigg(
        \frac{S^2}{1 + (\omega \tau_i)^2}
        + \frac{(1 - S^2_f)(\tau_f + \tau_i)\tau_f}{(\tau_f + \tau_i)^2 + (\omega \tau_f \tau_i)^2}       \\
        + \frac{(S^2_f - S^2)(\tau_s + \tau_i)\tau_s}{(\tau_s + \tau_i)^2 + (\omega \tau_s \tau_i)^2}
    \Bigg).
\end{multline}

\noindent where $S^2_f$ and $\tau_f$ are the amplitude and timescale of the faster of the two motions whereas $S^2_s$ and $\tau_s$ are those of the slower motion.  $S^2_f$ and $S^2_s$ are related by the formula $S^2 = S^2_f \cdot S^2_s$.

If these forms of the model-free spectral density functions are unfamiliar, that is because these are the numerically stabilised forms presented in \citet{dAuvergneGooley08a}.  The original model-free spectral density functions presented in \citet{LipariSzabo82a} and \citet{Clore90a} are not the most numerically stable form of these equations.  An important problem encountered in optimisation is round-off error in which machine precision influences the result of mathematical operations.  The double reciprocal $\tau^{-1} = \tau_m^{-1} + \tau_e^{-1}$ used in the equations are operations which are particularly susceptible to round-off error, especially when $\tau_e \ll \tau_m$.  By incorporating these reciprocals into the model-free spectral density functions and then simplifying the equations this source of round-off error can be eliminated, giving relax an edge over other model-free optimisation softwares.


% Brownian rotational diffusion.
\subsection{Brownian rotational diffusion}

\index{diffusion!Brownian|textbf}
In equations~\eqref{eq: J(w) model-free generic} and~\eqref{eq: J(w) model-free ext generic} the generic Brownian diffusion NMR correlation function presented in \citet{dAuvergne06} has been used.  This function is
\begin{equation} \label{eq: C(tau) generic}
    C(\tau) = \frac{1}{5} \sum_{i=-k}^k c_i \cdot e^{-\tau/\tau_i},
\end{equation}

\noindent where the summation index $i$ ranges over the number of exponential terms within the correlation function.  This equation is generic in that it can describe the diffusion\index{diffusion} of an ellipsoid, a spheroid, or a sphere.



% Diffusion as an ellipsoid.
\subsubsection{Diffusion as an ellipsoid}
\index{diffusion!ellipsoid (asymmetric)|textbf}

For the ellipsoid defined by the parameter set \{$\Diff_{iso}$, $\Diff_a$, $\Diff_r$, $\alpha$, $\beta$, $\gamma$\} the variable $k$ is equal to two and therefore the index $i \in \{-2, -1, 0, 1, 2\}$.  The geometric parameters \{$\Diff_{iso}$, $\Diff_a$, $\Diff_r$\} are defined as
\begin{subequations}
\begin{align}
    & \Diff_{iso} = \tfrac{1}{3} (\Diff_x + \Diff_y + \Diff_z ),   \label{eq: Diso ellipsoid def} \\
    & \Diff_a = \Diff_z - \tfrac{1}{2}(\Diff_x + \Diff_y),         \label{eq: Da ellipsoid def} \\
    & \Diff_r = \frac{\Diff_y - \Diff_x}{2\Diff_a},                \label{eq: Dr ellipsoid def}
\end{align}
\end{subequations}

\noindent and are constrained by
\begin{subequations}
\begin{align}
    0 & < \Diff_{iso} < \infty,                                                    \label{eq: Diso lim} \\
    0 & \le \Diff_a < \frac{\Diff_{iso}}{\tfrac{1}{3} + \Diff_r} \le 3\Diff_{iso}, \label{eq: Da lim} \\
    0 & \le \Diff_r \le 1.                                                         \label{eq: Dr lim}
\end{align}
\end{subequations}

\noindent The orientational parameters \{$\alpha$, $\beta$, $\gamma$\} are the Euler angles using the z-y-z rotation notation.


The five weights $c_i$ are defined as
\begin{subequations}
\begin{align}
    c_{-2} &= \tfrac{1}{4}(d - e),     \label{eq: ellipsoid c-2} \\
    c_{-1} &= 3\delta_y^2\delta_z^2,   \label{eq: ellipsoid c-1} \\
    c_{0}  &= 3\delta_x^2\delta_z^2,   \label{eq: ellipsoid c0} \\
    c_{1}  &= 3\delta_x^2\delta_y^2,   \label{eq: ellipsoid c1} \\
    c_{2}  &= \tfrac{1}{4}(d + e),     \label{eq: ellipsoid c2}
\end{align}
\end{subequations}

\noindent where
\begin{align}
    d &= 3 \left( \delta_x^4 + \delta_y^4 + \delta_z^4 \right) - 1, \label{eq: ellipsoid d} \\
    e &= \frac{1}{\mathfrak{R}} \bigg[ (1 + 3\Diff_r) \left(\delta_x^4 + 2\delta_y^2\delta_z^2\right)
        + (1 - 3\Diff_r) \left(\delta_y^4 + 2\delta_x^2\delta_z^2\right) - 2 \left(\delta_z^4 + 2\delta_x^2\delta_y^2\right) \bigg], \label{eq: ellipsoid e}
\end{align}

\noindent and where
\begin{equation}
    \mathfrak{R} = \sqrt{1 + 3\Diff_r^2}.
\end{equation}


The five correlation times $\tau_i$ are
\begin{subequations}
\begin{align}
    1/\tau_{-2} &= 6 \Diff_{iso} - 2\Diff_a\mathfrak{R},   \label{eq: ellipsoid tau-2} \\
    1/\tau_{-1} &= 6 \Diff_{iso} - \Diff_a (1 + 3\Diff_r), \label{eq: ellipsoid tau-1} \\
    1/\tau_{0}  &= 6 \Diff_{iso} - \Diff_a (1 - 3\Diff_r), \label{eq: ellipsoid tau0} \\
    1/\tau_{1}  &= 6 \Diff_{iso} + 2\Diff_a,               \label{eq: ellipsoid tau1} \\
    1/\tau_{2}  &= 6 \Diff_{iso} + 2\Diff_a\mathfrak{R}.   \label{eq: ellipsoid tau2}
\end{align}
\end{subequations}



% Diffusion as a spheroid.
\subsubsection{Diffusion as a spheroid}
\index{diffusion!spheroid (axially symmetric)|textbf}

The variable $k$ is equal to one in the case of the spheroid\index{diffusion!spheroid (axially symmetric)|textbf} defined by the parameter set \{$\Diff_{iso}$, $\Diff_a$, $\theta$, $\phi$\}, hence $i \in \{-1, 0, 1\}$.  The geometric parameters \{$\Diff_{iso}$, $\Diff_a$\} are defined as
\begin{subequations}
\begin{align}
    & \Diff_{iso} = \tfrac{1}{3} (\Diff_\Par + 2\Diff_\Per),   \label{eq: Diso spheroid def} \\
    & \Diff_a = \Diff_\Par - \Diff_\Per.                       \label{eq: Da spheroid def}
\end{align}
\end{subequations}

\noindent and are constrained by
\begin{subequations}
\begin{gather}
    0 < \Diff_{iso} < \infty, \\
    -\tfrac{3}{2} \Diff_{iso} < \Diff_a < 3\Diff_{iso}.
\end{gather}
\end{subequations}

\noindent The orientational parameters \{$\theta$, $\phi$\} are the spherical angles defining the orientation of the major axis of the diffusion frame within the lab frame.


The three weights $c_i$ are
\begin{subequations}
\begin{align}
    c_{-1} &= \tfrac{1}{4}(3\delta_z^2 - 1)^2, \label{eq: spheroid c-1} \\
    c_{0}  &= 3\delta_z^2(1 - \delta_z^2),     \label{eq: spheroid c0} \\
    c_{1}  &= \tfrac{3}{4}(\delta_z^2 - 1)^2.  \label{eq: spheroid c1}
\end{align}
\end{subequations}

The five correlation times $\tau_i$ are
\begin{subequations}
\begin{align}
    1/\tau_{-1} &= 6\Diff_{iso} - 2\Diff_a,    \label{eq: spheroid tau-1} \\
    1/\tau_{0}  &= 6\Diff_{iso} - \Diff_a,     \label{eq: spheroid tau0} \\
    1/\tau_{1}  &= 6\Diff_{iso} + 2\Diff_a.    \label{eq: spheroid tau1}
\end{align}
\end{subequations}



% Diffusion as a sphere.
\subsubsection{Diffusion as a sphere}
\index{diffusion!sphere (isotropic)|textbf}

In the situation of a molecule diffusing as a sphere\index{diffusion!sphere (isotropic)|textbf} either described by the single parameter $\tau_m$ or $\Diff_{iso}$, the variable $k$ is equal to zero.  Therefore $i \in \{0\}$.  The single weight $c_0$ is equal to one and the single correlation time $\tau_0$ is equivalent to the global tumbling time $\tau_m$ given by
\begin{equation} \label{eq: sphere tau0}
    1/\tau_m = 6\Diff_{iso}.
\end{equation}

\noindent This is diffusion equation presented in \citet{Bloembergen48}.


% The model-free models.
%~~~~~~~~~~~~~~~~~~~~~~~

\subsection{The model-free models}

Extending the list of models given in \citet{Mandel95, Fushman97, Orekhov99b, Korzhnev01, Zhuravleva04}, the models built into relax include
\begin{subequations}
\renewcommand{\theequation}{\theparentequation .\arabic{equation}}
\addtocounter{equation}{-1}
\begin{align}
 m0 &= \{\},                                   \label{model: m0} \\
 m1 &= \{S^2\},                                \label{model: m1} \\
 m2 &= \{S^2, \tau_e\},                        \label{model: m2} \\
 m3 &= \{S^2, R_{ex}\},                        \label{model: m3} \\
 m4 &= \{S^2, \tau_e, R_{ex}\},                \label{model: m4} \\
 m5 &= \{S^2, S^2_f, \tau_s\},                 \label{model: m5} \\
 m6 &= \{S^2, \tau_f, S^2_f, \tau_s\},         \label{model: m6} \\
 m7 &= \{S^2, S^2_f, \tau_s, R_{ex}\},         \label{model: m7} \\
 m8 &= \{S^2, \tau_f, S^2_f, \tau_s, R_{ex}\}, \label{model: m8} \\
 m9 &= \{R_{ex}\}.                             \label{model: m9}
\end{align}
\end{subequations}

\noindent The parameter $R_{ex}$ is scaled quadratically with field strength in these models as it is assumed to be fast.  In the set theory notation, the model-free model for the spin system $i$ is represented by the symbol $\Mfset_i$.  Through the addition of the local $\tau_m$ to each of these models, only the component of Brownian rotational diffusion experienced by the spin system is probed.  These models, represented in set notation by the symbol $\Localset_i$, are
\begin{subequations}
\renewcommand{\theequation}{\theparentequation .\arabic{equation}}
\addtocounter{equation}{-1}
\begin{align}
 tm0 &= \{\tau_m\},                                     \label{model: tm0} \\
 tm1 &= \{\tau_m, S^2\},                                \label{model: tm1} \\
 tm2 &= \{\tau_m, S^2, \tau_e\},                        \label{model: tm2} \\
 tm3 &= \{\tau_m, S^2, R_{ex}\},                        \label{model: tm3} \\
 tm4 &= \{\tau_m, S^2, \tau_e, R_{ex}\},                \label{model: tm4} \\
 tm5 &= \{\tau_m, S^2, S^2_f, \tau_s\},                 \label{model: tm5} \\
 tm6 &= \{\tau_m, S^2, \tau_f, S^2_f, \tau_s\},         \label{model: tm6} \\
 tm7 &= \{\tau_m, S^2, S^2_f, \tau_s, R_{ex}\},         \label{model: tm7} \\
 tm8 &= \{\tau_m, S^2, \tau_f, S^2_f, \tau_s, R_{ex}\}, \label{model: tm8} \\
 tm9 &= \{\tau_m, R_{ex}\}.                             \label{model: tm9}
\end{align}
\end{subequations}




% Model-free optimisation theory.
%~~~~~~~~~~~~~~~~~~~~~~~~~~~~~~~~

\subsection{Model-free optimisation theory}


% The model-free space.
\subsubsection{The model-free space}

The optimisation of the parameters of an arbitrary model is dependent on a function $f$ which takes the current parameter values $\theta \in \mathbb{R}^n$ and returns a single real value $f(\theta) \in \mathbb{R}$ corresponding to position $\theta$ in the $n$-dimensional space.  For it is that single value which is minimised as
\begin{equation}
 \hat\theta = \arg \min_\theta f(\theta),
\end{equation}

\noindent where $\hat\theta$ is the parameter vector which is equal to the argument which minimises the function $f(\theta)$.  In model-free analysis $f(\theta)$ is the chi-squared\index{chi-squared|textbf} equation
\begin{equation} \label{eq: chi2}
 \chi^2(\theta) = \sum_{i=1}^n \frac{(\Ri - \Ri(\theta))^2}{\sigma_i^2},
\end{equation}

\noindent where $i$ is the summation index, $\Ri$ is the experimental relaxation data which belongs to the data set R and includes the $\Rone$, $\Rtwo$, and NOE values at all field strengths, $\Ri(\theta)$ is the back calculated relaxation data belonging to the set R$(\theta)$, and $\sigma_i$ is the experimental error.  For the optimisation of the model-free parameters while the diffusion tensor is held fixed, the summation index ranges over the relaxation data of an individual spin.  If the diffusion parameters are optimised simultaneously with the model-free parameters the summation index ranges over all relaxation data of all selected spins of the macromolecule.

Given the current parameter values the model-free function provided to the algorithm will calculate the value of the model-free spectral density function $J(\omega)$ at the five frequencies which induce NMR relaxation by using Equations~\eqref{eq: J(w) model-free generic} and \eqref{eq: J(w) model-free ext generic}.  The theoretical $\Rone$, $\Rtwo$, and NOE values are then back-calculated using Equations~\eqref{eq: R1}, \eqref{eq: R2}, \eqref{eq: sigma_NOE}, and \eqref{eq: NOE}.  Finally, the chi-squared\index{chi-squared} value is calculated using Equation~\eqref{eq: chi2}.


% Topology of the space.
\subsubsection{Topology of the space}

The problem of finding the minimum is complicated by the fact that optimisation algorithms are blind to the curvature of the complete space.  Instead they rely on topological information about the current and, sometimes, the previous parameter positions in the space.  The techniques use this information to walk iteratively downhill to the minimum.  Very few optimisation algorithms rely solely on the function value, conceptually the height of the space, at the current position.  Most techniques also utilise the gradient at the current position.  Although symbolically complex in the case of model-free analysis, the gradient can simply be calculated as the vector of first partial derivatives of the chi-squared\index{chi-squared} equation with respect to each model-free parameter.  The gradient is supplied as a second function to the algorithm which is then utilised in diverse ways by different optimisation techniques.  The function value together with the gradient can be combined to construct a linear or planar description of the space at the current parameter position by first-order Taylor series approximation
\begin{equation} \label{eq: linear model}
 f(\theta_k + x) \approx f_k  +  x^T \nabla f_k,
\end{equation}

\noindent where $f_k$ is the function value at the current parameter position $\theta_k$, $\nabla f_k$ is the gradient at the same position, and $x$ is an arbitrary vector.  By accumulating information from previous parameter positions a more comprehensive geometric description of the curvature of the space can be exploited by the algorithm for more efficient optimisation.

The best and most comprehensive description of the space is given by the quadratic approximation of the topology which is generated from the combination of the function value, the gradient, and the Hessian.  From the second-order Taylor series expansion the quadratic model of the space is
\begin{equation} \label{eq: quadratic model}
 f(\theta_k + x) \approx f_k  +  x^T \nabla f_k  +  \tfrac{1}{2} x^T \nabla^2 f_k x,
\end{equation}

\noindent where $\nabla^2 f_k$ is the Hessian, which is the symmetric matrix of second partial derivatives of the function, at the position $\theta_k$.  As the Hessian is computationally expensive a number of optimisation algorithms try to approximate it.

To produce the gradient and Hessian required for model-free optimisation a large chain of first and second partial derivatives needs to be calculated.  Firstly the partial derivatives of the spectral density functions \eqref{eq: J(w) model-free generic} and \eqref{eq: J(w) model-free ext generic} are necessary.  Then the partial derivatives of the relaxation equations~\eqref{eq: R1} to~\eqref{eq: sigma_NOE} followed by the NOE equation~\eqref{eq: NOE} are needed.  Finally the partial derivative of the chi-squared\index{chi-squared} formula~\eqref{eq: chi2} is required.  These first and second partial derivatives, as well as those of the components of the Brownian diffusion correlation function for non-isotropic tumbling, are presented in Chapter~\ref{ch: values, gradients, and Hessians}.



% Optimisation algorithms.
\subsubsection{Optimisation algorithms}

Prior to minimisation, all optimisation algorithms investigated require a starting position within the model-free space.  This initial parameter vector is found by employing a coarse grid search -- chi-squared\index{chi-squared} values at regular positions spanning the space are calculated and the grid point with the lowest value becomes the starting position.  The grid search itself is an optimisation technique.  As it is computationally expensive the number of grid points needs to be kept to a minimum.  Hence the initial parameter values are a rough and imprecise approximation of the local minimum.  Due to the complexity of the curvature of the model-free space, the grid point with the lowest chi-squared\index{chi-squared} value may in fact be on the opposite side of the space to the local minimum.

Once the starting position has been determined by the grid search the optimisation algorithm can be executed.  The number of algorithms developed within the mathematical field of optimisation is considerable.  They can nevertheless be grouped into one of a small number of major categories based on the fundamental principles of the technique.  These include the line search methods, the trust region methods, and the conjugate gradient methods.  For more details on the algorithms described below see \citet{NocedalWright99}.



% Line search methods.
\subsubsection{Line search methods}

The defining characteristic of a line search algorithm is to choose a search direction $p_k$ and then to find the minimum along that vector starting from $\theta_k$ \citep{NocedalWright99}.  The distance travelled along $p_k$ is the step length $\alpha_k$ and the parameter values for the next iteration are
\begin{equation}
 \theta_{k+1} = \theta_k + \alpha_k p_k.
\end{equation}

\noindent  The line search algorithm determines the search direction $p_k$ whereas the value of $\alpha_k$ is found using an auxiliary step-length selection algorithm.

One of the simplest line search methods is the steepest descent\index{minimisation algorithm!steepest descent|textbf} algorithm.  The search direction is simply the negative gradient, $p_k = -\nabla f_k$, and hence the direction of maximal descent is always followed.  This method is inefficient -- the linear rate of convergence requires many iterations of the algorithm to reach the minimum and it is susceptible to being trapped on saddle points within the space.

The coordinate descent\index{minimisation algorithm!coordinate descent|textbf} algorithms are a simplistic group of line search methods whereby the search directions alternate between vectors parallel to the parameter axes.  For the back-and-forth coordinate descent the search directions cycle in one direction and then back again.  For example for a three parameter model the search directions cycle $\theta_1, \theta_2, \theta_3, \theta_2, \theta_1, \theta_2, \hdots$, which means that each parameter of the model is optimised one by one.  The method becomes less efficient when approaching the minimum as the step length $\alpha_k$ continually decreases (ibid.).

The quasi-Newton methods begin with an initial guess of the Hessian and update it at each iteration using the function value and gradient.  Therefore the benefits of using the quadratic model of \eqref{eq: quadratic model} are obtained without calculating the computationally expensive Hessian.  The Hessian approximation $B_k$ is updated using various formulae, the most common being the BFGS\index{minimisation algorithm!BFGS|textbf} formula \citep{Broyden70,Fletcher70,Goldfarb70,Shanno70}.  The search direction is given by the equation $p_k = -B_k^{-1} \nabla f_k$.  The quasi-Newton algorithms can attain a superlinear rate of convergence, being superior to the steepest descent or coordinate descent methods.

The most powerful line search method when close to the minimum is the Newton\index{minimisation algorithm!Newton|textbf} search direction
\begin{equation} \label{eq: Newton dir}
 p_k = - \nabla^2 f_k^{-1} \nabla f_k.
\end{equation}

\noindent This direction is obtained from the derivative of \eqref{eq: quadratic model} which is assumed to be zero at the minimum of the quadratic model.  The vector $p_k$ points from the current position to the exact minimum of the quadratic model of the space.  The rate of convergence is quadratic, being superior to both linear and superlinear convergence.  The technique is computationally expensive due to the calculation of the Hessian.  It is also susceptible to failure when optimisation commences from distant positions in the space as the Hessian may not be positive definite and hence not convex, a condition required for the search direction both to point downhill and to be reasonably oriented.  In these cases the quadratic model is a poor description of the space.

A practical Newton algorithm which is robust for distant starting points is the Newton conjugate gradient method (Newton-CG\index{minimisation algorithm!Newton-CG|textbf}).  This line search method, which is also called the truncated Newton algorithm, finds an approximate solution to Equation~\eqref{eq: Newton dir} by using a conjugate gradient (CG) sub-algorithm.  Retaining the performance of the pure Newton algorithm, the CG sub-algorithm guarantees that the search direction is always downhill as the method terminates when negative curvature is encountered.  This algorithm is similar to the Newton-Raphson-CG algorithm implemented within Dasha\index{software!Dasha}.  Newton optimisation is sometimes also known as the Newton-Raphson algorithm and, as documented in the source code, the Newton algorithm in Dasha is coupled to a conjugate gradient algorithm.  The auxiliary ste\mbox{p-l}ength selection algorithm in Dasha\index{software!Dasha} is undocumented and may not be employed.

Once the search direction has been determined by the above algorithms the minimum along that direction needs to be determined.  Not to be confused with the methodology for determining the search direction $p_k$, the line search itself is performed by an auxiliary step-length selection algorithm to find the value $\alpha_k$.  A number of step-length selection methods can be used to find a minimum along the line $\theta_k + \alpha_k p_k$, although only two will be investigated.  The first is the backtracking line search of \citet{NocedalWright99}.  This method is inexact -- it takes a starting step length $\alpha_k$ and decreases the value until a sufficient decrease in the function is found.  The second is the line search method of \citet{MoreThuente94}.  Designed to be robust, the MT algorithm finds the exact minimum along the search direction and guarantees sufficient decrease.



% Trust region methods.
\subsubsection{Trust region methods}

In the trust region class of algorithms the curvature of the space is modelled quadratically by \eqref{eq: quadratic model}.  This model is assumed to be reliable only within a region of trust defined by the inequality $\lVert p \rVert \leqslant \Delta_k$ where $p$ is the step taken by the algorithm and $\Delta_k$ is the radius of the $n$-dimensional sphere of trust \citep{NocedalWright99}.  The solution sought for each iteration of the algorithm is
\begin{equation} \label{eq: trust region}
 \min_{p \in \mathbb{R}^n} m_k(p) = f_k  +  p^{T} \nabla f_k  +  \tfrac{1}{2} p^{T} B_k p,  \qquad \textrm{s.t. } \lVert p \rVert \leqslant \Delta_k,
\end{equation}

\noindent where $m_k(p)$ is the quadratic model, $B_k$ is a positive definite matrix which can be the true Hessian as in the Newton model or an approximation such as the BFGS\index{minimisation algorithm!BFGS} matrix, and $\lVert p \rVert$ is the Euclidean norm of $p$.  The trust region radius $\Delta_k$ is modified dynamically during optimisation -- if the quadratic model is found to be a poor representation of the space the radius is decreased whereas if the quadratic model is found to be reasonable the radius is increased to allow larger, more efficient steps to be taken.

The Cauchy point\index{minimisation algorithm!Cauchy point|textbf} algorithm is similar in concept to the steepest descent\index{minimisation algorithm!steepest descent} line search algorithm.  The Cauchy point is the point lying on the gradient which minimises the quadratic model subject to the step being within the trust region.  By iteratively finding the Cauchy point the local minimum can be found.  The convergence of the technique is inefficient, being similar to that of the steepest descent algorithm.

In changing the trust region radius the exact solutions to \eqref{eq: trust region} map out a curved trajectory which starts parallel to the gradient for small radii.  The end of the trajectory, which occurs for radii greater than the step length, is the bottom of the quadratic model.  The dogleg\index{minimisation algorithm!dogleg|textbf} algorithm attempts to follow a similar path by first finding the minimum along the gradient and then finding the minimum along a trajectory from the current point to the bottom of the quadratic model.  The minimum along the second path is either the trust region boundary or the quadratic solution.  The matrix $B_k$ of \eqref{eq: trust region} can be the BFGS matrix, the unmodified Hessian, or a Hessian modified to be positive definite.

Another trust region algorithm is Steihaug's\index{minimisation algorithm!CG-Steihaug|textbf} modified conjugate gradient approach \citep{Steihaug83}.  For each step $k$ an iterative technique is used which is almost identical to the standard conjugate gradient procedure except for two additional termination conditions.  The first is if the next step is outside the trust region, the second is if a direction of zero or negative curvature is encountered.

An almost exact solution to \eqref{eq: trust region} can be found using an algorithm described in \citet{NocedalWright99}.  This exact trust region\index{minimisation algorithm!exact trust region|textbf} algorithm aims to precisely find the minimum of the quadratic model $m_k$ of the space within the trust region $\Delta_k$.  Any matrix $B_k$ can be used to construct the quadratic model.  However, the technique is computationally expensive.



% Conjugate gradient methods.
\subsubsection{Conjugate gradient methods}

The conjugate gradient algorithm (CG) was originally designed as a mathematical technique for solving a large system of linear equations \citet{HestenesStiefel52}, but was later adapted to solving nonlinear optimisation problems \citep{FletcherReeves64}.  The technique loops over a set of directions $p_0$, $p_1$, $\hdots$, $p_n$ which are all conjugate to the Hessian \citep{NocedalWright99}, a property defined as
\begin{equation}
 p_i^T \nabla^2 f_k p_j = 0,  \qquad \textrm{for all } i \ne j.
\end{equation}

\noindent By performing line searches over all directions $p_j$ the solution to the quadratic model \eqref{eq: quadratic model} of the position $\theta_k$ will be found in $n$ or less iterations of the CG algorithm where $n$ is the total number of parameters in the model.  The technique performs well on large problems with many parameters as no matrices are calculated or stored.  The algorithms perform better than the steepest descent method and preconditioning of the system is used to improve optimisation.  A number of preconditioned techniques will be investigated including the Fletcher-Reeves\index{minimisation algorithm!Fletcher-Reeves|textbf} algorithm which was the original conjugate gradient optimisation technique \citep{FletcherReeves64}, the Polak-Ribi\`ere\index{minimisation algorithm!Polak-Ribi\`ere|textbf} method \citep{PolakRibiere69}, a modified Polak-Ribi\`ere method called the Polak-Ribi\`ere +\index{minimisation algorithm!Polak-Ribi\`ere~+|textbf} method \citep{NocedalWright99}, and the Hestenes-Stiefel\index{minimisation algorithm!Hestenes-Stiefel|textbf} algorithm which originates from a formula in \citet{HestenesStiefel52}.  As a line search is performed to find the minimum along each conjugate direction both the backtracking and Mor\'e and Thuente auxiliary step-length selection algorithms will be tested with the CG algorithms.



% Hessian modifications.
\subsubsection{Hessian modifications}

The Newton search direction, used in both the line search and trust region methods, is dependent on the Hessian being positive definite for the quadratic model to be convex so that the search direction points sufficiently downhill.  This is not always the case as saddle points and other non-quadratic features of the space can be problematic.  Two classes of algorithms can be used to handle this situation.  The first involves using the conjugate gradient method as a sub-algorithm for solving the Newton problem for the step $k$.  The Newton-CG\index{minimisation algorithm!Newton-CG} line search algorithm described above is one such example.  The second class involves modifying the Hessian prior to, or at the same time as, finding the Newton step to guarantee that the replacement matrix $B_k$ is positive definite.  The convexity of $B_k$ is ensured by its eigenvalues all being positive.  The performance of two of these methods within the model-free space will be investigated.

The first modification uses the Cholesky factorisation of the matrix $B_k$, initialised to the true Hessian, to test for convexity (Algorithm 6.3 of \citet{NocedalWright99}).  If factorisation fails the matrix is not positive definite and a constant $\tau_k$ times the identity matrix $I$ is then added to $B_k$.  The constant originates from the Robbins norm of the Hessian $\lVert \nabla^2 f_k \rVert_F$ and is steadily increased until the factorisation is successful.  The resultant Cholesky lower triangular matrix $L$ can then be used to find the approximate Newton direction.  If $\tau_k$ is too large the convergence of this technique can approach that of the steepest descent\index{minimisation algorithm!steepest descent} algorithm.

The second method is the Gill, Murray, and Wright (GMW) algorithm \citep{GMW81} which modifies the Hessian during the execution of the Cholesky factorisation $\nabla^2 f_k = LIL^T$, where $L$ is a lower triangular matrix and $D$ is a diagonal matrix.  Only a single factorisation is required.  As rows and columns are interchanged during the algorithm the technique may be slow for large problems such as the optimisation of the model-free parameters of all spins together with the diffusion tensor parameters.  The rate of convergence of the technique is quadratic.



% Other methods.
\subsubsection{Other methods}

Two other optimisation algorithms which cannot be classified within line search, trust region, or conjugate gradient categories will also be investigated.  The first is the well known simplex\index{minimisation algorithm!simplex|textbf} optimisation algorithm.  The technique is often used as the only the function value is employed and hence the derivation of the gradient and Hessian can be avoided.  The simplex is created as an $n$-dimensional geometric object with $n+1$ vertices.  The first vertex is the starting position.  Each of the other $n$ vertices are created by shifting the starting position by a small amount parallel to one of unit vectors defining the coordinate system of the space.  Four simple rules are used to move the simplex through the space: reflection, extension, contraction, and a shrinkage of the entire simplex.  The result of these movements is that the simplex moves in an ameoboid-like fashion downhill, shrinking to pass through tight gaps and expanding to quickly move through non-convoluted space, eventually finding the minimum.

Key to these four movements is the pivot point, the centre of the face created by the $n$ vertices with the lowest function values.  The first movement is a reflection -- the vertex with the greatest function value is reflected through the pivot point on the opposite face of the simplex.  If the function value at this new position is less than all others the simplex is allowed to extend -- the point is moved along the line to twice the distance between the current position and the pivot point.  Otherwise if the function value is greater than the second highest value but less than the highest value, the reflected simplex is contracted.  The reflected point is moved to be closer to the simplex, its position being half way between the reflected position and the pivot point.  Otherwise if the function value at the reflected point is greater than all other vertices, then the original simplex is contracted -- the highest vertex is moved to a position half way between the current position and the pivot point.  Finally if none of these four movements yield an improvement, then the simplex is shrunk halfway towards the vertex with the lowest function value.

The other algorithm is the commonly used Levenberg-Marquardt\index{minimisation algorithm!Levenberg-Marquardt|textbf} algorithm \citep{Levenberg44,Marquardt63} which is implemented in Modelfree4\index{software!Modelfree}, Dasha\index{software!Dasha}, and Tensor2\index{software!Tensor}.  This technique is designed for least-squares problems to which the chi-squared\index{chi-squared} equation \eqref{eq: chi2} belongs.  The key to the algorithm is the replacement of the Hessian with the Levenberg-Marquardt matrix $J^T J + \lambda I$, where $J$ is the Jacobian of the system calculated as the matrix of partial derivatives of the residuals, $\lambda > 0$ is a factor related to the trust-region radius, and $I$ is the identity matrix.  The algorithm is conceptually allied to the trust region methods and its performance varies finely between that of the steepest descent and the pure Newton step.  When far from the minimum $\lambda$ is large and the algorithm takes steps close to the gradient; when in vicinity of the minimum $\lambda$ heads towards zero and the steps taken approximate the Newton direction.  Hence the algorithm avoids the problems of the Newton\index{minimisation algorithm!Newton} algorithm when non-convex curvature is encountered and approximates the Newton step in convex regions of the space.



% Constraint algorithms.
\subsubsection{Constraint algorithms}

To guarantee that the minimum will still be reached the implementation of constraints limiting the parameter values together with optimisation algorithms is not a triviality.  For this to occur the space and its boundaries must remain smooth thereby allowing the algorithm to move along the boundary to either find the minimum along the limit or to slide along the limit and then move back into the centre of the constrained space once the curvature allows it.  One of the most powerful approaches is the Method of Multipliers \citep{NocedalWright99}, also known as the Augmented Lagrangian.  Instead of a single optimisation the algorithm is iterative with each iteration consisting of an independent unconstrained minimisation on a sequentially modified space.  When inside the limits the function value is unchanged but when outside a penalty, which is proportional to the distance outside the limit, is added to the function value.  This penalty, which is based on the Lagrange multipliers, is smooth and hence the gradient and Hessian are continuous at and beyond the constraints.  For each iteration of the Method of Multipliers the penalty is increased until it becomes impossible for the parameter vector to be in violation of the limits.  This approach allows the parameter vector $\theta$ outside the limits yet the successive iterations ensure that the final results will not be in violation of the constraint.

For inequality constraints, each iteration of the Method of Multipliers attempts to solve the quadratic sub-problem
\begin{equation} \label{eq: Augmented Lagrangian}
    \min_\theta \mathfrak{L}_A(\theta, \lambda^k; \mu_k) \stackrel{\mathrm{def}}{=} f(\theta) + \sum_{i \in \mathfrak{I}} \Psi(c_i(\theta), \lambda_i^k; \mu_k),
\end{equation}

\noindent where the function $\Psi$ is defined as
\begin{equation}
    \Psi(c_i(\theta), \lambda^k; \mu_k) = \begin{cases}
        -\lambda^k c_i(\theta) + \frac{1}{2\mu_k} c_i^2(\theta) & \textrm{if } c_i(\theta) - \mu_k \lambda^k \leqslant 0, \\
        -\frac{\mu_k}{2} (\lambda^k)^2 & \textrm{otherwise}.
    \end{cases}
\end{equation}

\noindent  In \eqref{eq: Augmented Lagrangian}, $\theta$ is the parameter vector; $\mathfrak{L}_A$ is the Augmented Lagrangian function; $k$ is the current iteration of the Method of Multipliers; $\lambda^k$ are the Lagrange multipliers which are positive factors such that, at the minimum $\hat\theta$, $\nabla f(\hat\theta) = \lambda_i \nabla c_i(\hat\theta)$; $\mu_k > 0$ is the penalty parameter which decreases to zero as $k \to \infty$; $\mathfrak{I}$ is the set of inequality constraints; and $c_i(\theta)$ is an individual constraint value.  The Lagrange multipliers are updated using the formula
\begin{equation}
    \lambda_i^{k+1} = \max(\lambda_i^k - c_i(\theta)/\mu_k, 0), \qquad \textrm{for all } i \in \mathfrak{I}.
\end{equation}

The gradient of the Augmented Lagrangian is
\begin{equation}
    \nabla \mathfrak{L}_A(\theta, \lambda^k; \mu_k) = 
        \nabla f(\theta)
        - \sum_{i \in \mathfrak{I} | c_i(\theta) \leqslant \mu_k \lambda_i^k}
            \left( \lambda_i^k - \frac{c_i(\theta)}{\mu_k} \right) \nabla c_i(\theta),
\end{equation}

\noindent and the Hessian is
\begin{equation}
    \nabla^2 \mathfrak{L}_A(\theta, \lambda^k; \mu_k) = 
        \nabla^2 f(\theta)
        + \sum_{i \in \mathfrak{I} | c_i(\theta) \leqslant \mu_k \lambda_i^k}
            \left[
                \frac{1}{\mu_k} \nabla c_i^2(\theta)
                - \left( \lambda_i^k - \frac{c_i(\theta)}{\mu_k} \right) \nabla^2 c_i(\theta)
            \right].
\end{equation}

The Augmented Lagrangian algorithm can accept any set of three arbitrary constraint functions $c(\theta)$, $\nabla c(\theta)$, and $\nabla^2 c(\theta)$.  When given the current parameter values $c(\theta)$ returns a vector of constraint values whereby each position corresponds to one of the model parameters.  The constraint is defined as $c_i \geqslant 0$.  The function $\nabla c(\theta)$ returns the matrix of constraint gradients and $\nabla^2 c(\theta)$ is the constraint Hessian function which should return the 3D matrix of constraint Hessians.

A more specific set of constraints accepted by the Method of Multipliers are bound constraints.  These are defined by the function
\begin{equation}
    l \leqslant \theta \leqslant u,
\end{equation}

\noindent where $l$ and $u$ are the vectors of lower and upper bounds respectively and $\theta$ is the parameter vector.  For example for model-free model $m4$ to place lower and upper bounds on the order parameter and lower bounds on the correlation time and chemical exchange parameters, the vectors are
\begin{equation}
    \begin{pmatrix}
        0 \\
        0 \\
        0 \\
    \end{pmatrix}
    \leqslant
    \begin{pmatrix}
        S^2 \\
        \tau_e \\
        R_{ex} \\
    \end{pmatrix}
    \leqslant
    \begin{pmatrix}
        1 \\
        \infty \\
        \infty \\
    \end{pmatrix}.
\end{equation}

The default setting in the program relax\index{software!relax} is to use linear constraints which are defined as
\begin{equation} \label{eq: linear constraint}
    A \cdot \theta \geqslant b,
\end{equation}

\noindent where $A$ is an $m \times n$ matrix where the rows are the transposed vectors $a_i$ of length $n$; the elements of $a_i$ are the coefficients of the model parameters; $\theta$ is the vector of model parameters of dimension $n$; $b$ is the vector of scalars of dimension $m$; $m$ is the number of constraints; and $n$ is the number of model parameters.  For model-free analysis, linear constraints are the most useful type of constraint as the correlation time $\tau_f$ can be restricted to being less than $\tau_s$ by using the inequality $\tau_s - \tau_f \geqslant 0$.

In rearranging \eqref{eq: linear constraint} the linear constraint function $c(\theta)$ returns the vector $A \cdot \theta - b$.  Because of the linearity of the constraints the gradient and Hessian are greatly simplified.  The gradient $\nabla c(\theta)$ is simply the matrix $A$ and the Hessian $\nabla^2 c(\theta)$ is zero.  For the parameters specific to individual spins the linear constraints in the notation of \eqref{eq: linear constraint} are
\begin{equation}
    \begin{pmatrix}
        1 & 0 & 0 & 0 & 0 & 0 & 0 & 0 & 0 \\
        1 & 0 & 0 & 0 & 0 & 0 & 0 & 0 & 0 \\
        0 & 1 & 0 & 0 & 0 & 0 & 0 & 0 & 0 \\
        0 &-1 & 0 & 0 & 0 & 0 & 0 & 0 & 0 \\
        0 & 0 & 1 & 0 & 0 & 0 & 0 & 0 & 0 \\
        0 & 0 &-1 & 0 & 0 & 0 & 0 & 0 & 0 \\
        1 & 1 & 0 & 0 & 0 & 0 & 0 & 0 & 0 \\
        1 & 0 & 1 & 0 & 0 & 0 & 0 & 0 & 0 \\
        0 & 0 & 0 & 1 & 0 & 0 & 0 & 0 & 0 \\
        0 & 0 & 0 & 0 & 1 & 0 & 0 & 0 & 0 \\
        0 & 0 & 0 & 0 & 0 & 1 & 0 & 0 & 0 \\
        0 & 0 & 0 & 0 &-1 & 1 & 0 & 0 & 0 \\
        0 & 0 & 0 & 0 & 0 & 0 & 1 & 0 & 0 \\
        0 & 0 & 0 & 0 & 0 & 0 & 0 & 1 & 0 \\
        0 & 0 & 0 & 0 & 0 & 0 & 0 &-1 & 0 \\
        0 & 0 & 0 & 0 & 0 & 0 & 0 & 0 & 1 \\
        0 & 0 & 0 & 0 & 0 & 0 & 0 & 0 &-1 \\
    \end{pmatrix}
    \cdot
    \begin{pmatrix}
        S^2 \\
        S^2_f \\
        S^2_s \\
        \tau_e \\
        \tau_f \\
        \tau_s \\
        R_{ex} \\
         r \\
        CSA \\
    \end{pmatrix}
    \geqslant
    \begin{pmatrix}
        0 \\
        -1 \\
        0 \\
        -1 \\
        0 \\
        -1 \\
        0 \\
        0 \\
        0 \\
        0 \\
        0 \\
        0 \\
        0 \\
        0.9e^{-10} \\
        2e^{-10} \\
        300e^{-6} \\
        0 \\
    \end{pmatrix}.
\end{equation}

\noindent  Through the isolation of each individual element, the constraints can be see to be equivalent to
\begin{subequations}
\begin{gather} 
    0 \leqslant S^2 \leqslant 1, \\
    0 \leqslant S^2_f \leqslant 1, \\
    0 \leqslant S^2_s \leqslant 1, \\
    S^2 \leqslant S^2_f, \\
    S^2 \leqslant S^2_s, \\
    \tau_e \geqslant 0, \\
    \tau_f \geqslant 0, \\
    \tau_s \geqslant 0, \\
    \tau_s \geqslant 0, \\
    \tau_f \leqslant \tau_s, \\
    R_{ex} \geqslant 0, \\
    0.9e^{-10} \leqslant r \leqslant 2e^{-10}, \\
    -300e^{-6} \leqslant CSA \leqslant 0.
\end{gather} 
\end{subequations}

To prevent the computationally expensive optimisation of failed models in which the internal correlation times minimise to infinity \citep{dAuvergneGooley06}, the constraint $\tau_e, \tau_f, \tau_s \leqslant 2\tau_m$ was implemented.  When the global correlation time is fixed the constraints in the matrix notation of \eqref{eq: linear constraint} are
\begin{equation}
    \begin{pmatrix}
        -1 &  0 &  0 \\
         0 & -1 &  0 \\
         0 &  0 & -1 \\
    \end{pmatrix}
    \cdot
    \begin{pmatrix}
        \tau_e \\
        \tau_f \\
        \tau_s \\
    \end{pmatrix}
    \geqslant
    \begin{pmatrix}
        -2\tau_m \\
        -2\tau_m \\
        -2\tau_m \\
    \end{pmatrix}.
\end{equation}

\noindent  However when the global correlation time $\tau_m$ is one of the parameters being optimised the constraints become
\begin{equation}
    \begin{pmatrix}
        2 & -1 &  0 &  0 \\
        2 &  0 & -1 &  0 \\
        2 &  0 &  0 & -1 \\
    \end{pmatrix}
    \cdot
    \begin{pmatrix}
        \tau_m \\
        \tau_e \\
        \tau_f \\
        \tau_s \\
    \end{pmatrix}
    \geqslant
    \begin{pmatrix}
        0 \\
        0 \\
        0 \\
    \end{pmatrix}.
\end{equation}

For the parameters of the diffusion tensor the constraints utilised are
\begin{subequations}
\begin{gather} 
    0 \leqslant \tau_m \leqslant 200.0e^{-9}, \\
    \Diff_a \geqslant 0, \\
    0 \leqslant \Diff_r \leqslant 1,
\end{gather} 
\end{subequations}

\noindent  which in the matrix notation of \eqref{eq: linear constraint} become
\begin{equation}
    \begin{pmatrix}
         1 &  0 &  0 \\
        -1 &  0 &  0 \\
         0 &  1 &  0 \\
         0 &  0 &  1 \\
         0 &  0 & -1 \\
    \end{pmatrix}
    \cdot
    \begin{pmatrix}
        \tau_m \\
        \Diff_a \\
        \Diff_r \\
    \end{pmatrix}
    \geqslant
    \begin{pmatrix}
        0 \\
        -200.0e^{-9} \\
        0 \\
        0 \\
        -1 \\
    \end{pmatrix}.
\end{equation}

\noindent  The upper limit of 200~ns on $\tau_m$ prevents the parameter from heading towards infinity when model failure occurs (see \citet{dAuvergneGooley06}).  This can significantly decrease the computation time.  To isolate the prolate spheroid\index{diffusion!spheroid (axially symmetric)} the constraint
\begin{equation}
    \begin{pmatrix}
         1 \\
    \end{pmatrix}
    \cdot
    \begin{pmatrix}
        \Diff_a \\
    \end{pmatrix}
    \geqslant
    \begin{pmatrix}
        0 \\
    \end{pmatrix},
\end{equation}

\noindent is used whereas to isolate the oblate spheroid\index{diffusion!spheroid (axially symmetric)} the constraint used is
\begin{equation}
    \begin{pmatrix}
         -1 \\
    \end{pmatrix}
    \cdot
    \begin{pmatrix}
        \Diff_a \\
    \end{pmatrix}
    \geqslant
    \begin{pmatrix}
        0 \\
    \end{pmatrix}.
\end{equation}

Dependent on the model optimised, the matrix $A$ and vector $b$ are constructed from combinations of the above linear constraints.




% Diagonal scaling.
\subsubsection{Diagonal scaling} \label{sect: diagonal scaling}

Model scaling can have a significant effect on the optimisation algorithm -- a poorly scaled model can cause certain techniques to fail.  When two parameters of the model lie on very different numeric scales the model is said to be poorly scaled.  For example in model-free analysis the order of magnitude of the order parameters is one whereas for the internal correlation times the order of magnitude is between $1e^{-12}$ to $1e^{-8}$.  Most effected are the trust region algorithms -- the multidimensional sphere of trust will either be completely ineffective against the correlation time parameters or severely restrict optimisation in the order parameter dimensions.  In model-free analysis the significant scaling disparity can even cause failure of optimisation due to amplified effects of machine precision.  Therefore the model parameters need to be scaled.

This can be done by supplying the optimisation algorithm with the scaled rather than unscaled parameters.  When the chi-squared\index{chi-squared} function, gradient\index{chi-squared gradient}, and Hessian\index{chi-squared Hessian} are called the vector is then premultiplied with a diagonal matrix in which the diagonal elements are the scaling factors.  For the model-free analysis the scaling factor of one was used for the order parameter and a scaling factor of $1e^{-12}$ was used for the correlation times.  The $R_{ex}$ parameter was scaled to be the chemical exchange rate of the first field strength.  The scaling matrix for the parameters \{$S^2$, $S^2_f$, $S^2_s$, $\tau_e$, $\tau_f$, $\tau_s$, $R_{ex}$, $r$, $CSA$\} of individual spins is
\begin{equation}
    \begin{pmatrix}
        1 &  0 &  0 &  0 &  0 &  0 &  0 &  0 &  0 \\
        0 &  1 &  0 &  0 &  0 &  0 &  0 &  0 &  0 \\
        0 &  0 &  1 &  0 &  0 &  0 &  0 &  0 &  0 \\
        0 &  0 &  0 &  1e^{-12} &  0 &  0 &  0 &  0 &  0 \\
        0 &  0 &  0 &  0 &  1e^{-12} &  0 &  0 &  0 &  0 \\
        0 &  0 &  0 &  0 &  0 &  1e^{-12} &  0 &  0 &  0 \\
        0 &  0 &  0 &  0 &  0 &  0 &  (2\pi \omega_H)^{-2} &  0 &  0 \\
        0 &  0 &  0 &  0 &  0 &  0 &  0 &  1e^{-10} &  0 \\
        0 &  0 &  0 &  0 &  0 &  0 &  0 &  0 &  1e^{-4} \\
    \end{pmatrix}.
\end{equation}

\noindent  For the ellipsoidal\index{diffusion!ellipsoid (asymmetric)} diffusion parameters \{$\tau_m$, $\Diff_a$, $\Diff_r$, $\alpha$, $\beta$, $\gamma$\} the scaling matrix is
\begin{equation}
    \begin{pmatrix}
        1e^{-12} &  0 &  0 &  0 &  0 &  0 \\
        0 &  1e^7 &  0 &  0 &  0 &  0 \\
        0 &  0 &  1 &  0 &  0 &  0 \\
        0 &  0 &  0 &  1 &  0 &  0 \\
        0 &  0 &  0 &  0 &  1 &  0 \\
        0 &  0 &  0 &  0 &  0 &  1 \\
    \end{pmatrix}.
\end{equation}

\noindent  For the spheroidal\index{diffusion!spheroid (axially symmetric)} diffusion parameters \{$\tau_m$, $\Diff_a$, $\theta$, $\phi$\} the scaling matrix is
\begin{equation}
    \begin{pmatrix}
        1e^{-12} &  0 &  0 &  0 \\
        0 &  1e^7 &  0 &  0 \\
        0 &  0 &  1 &  0 \\
        0 &  0 &  0 &  1 \\
    \end{pmatrix}.
\end{equation}




% Optimisation of a single model-free model.
%%%%%%%%%%%%%%%%%%%%%%%%%%%%%%%%%%%%%%%%%%%%

\section{Optimisation of a single model-free model}\label{sect: single mf model}


% The sample script.
%~~~~~~~~~~~~~~~~~~~

\subsection{The sample script}

The sample script which demonstrates the optimisation of model-free model $m4$ which consists of the parameters \{$S^2$, $\tau_e$, $R_{ex}$\} is \file{model\_free/single\_model.py}.  The text of the script is:

\begin{exampleenv}
\# Script for model-free analysis. \\
 \\
\# Create the data pipe. \\
name = `m4' \\
pipe.create(name, `mf') \\
 \\
\# Set up the 15N spins. \\
sequence.read(`noe.500.out', res\_num\_col=1, res\_name\_col=2) \\
spin.name(`N') \\
spin.element(element=`N', spin\_id=`@N') \\
spin.isotope(`15N', spin\_id=`@N') \\
 \\
\# Load the relaxation data. \\
relax\_data.read(ri\_id=`R1\_600',  ri\_type=`R1',  frq=600.0*1e6, file=`r1.600.out', res\_num\_col=1, data\_col=3, error\_col=4) \\
relax\_data.read(ri\_id=`R2\_600',  ri\_type=`R2',  frq=600.0*1e6, file=`r2.600.out', res\_num\_col=1, data\_col=3, error\_col=4) \\
relax\_data.read(ri\_id=`NOE\_600', ri\_type=`NOE', frq=600.0*1e6, file=`noe.600.out', res\_num\_col=1, data\_col=3, error\_col=4) \\
relax\_data.read(ri\_id=`R1\_500',  ri\_type=`R1',  frq=500.0*1e6, file=`r1.500.out', res\_num\_col=1, data\_col=3, error\_col=4) \\
relax\_data.read(ri\_id=`R2\_500',  ri\_type=`R2',  frq=500.0*1e6, file=`r2.500.out', res\_num\_col=1, data\_col=3, error\_col=4) \\
relax\_data.read(ri\_id=`NOE\_500', ri\_type=`NOE', frq=500.0*1e6, file=`noe.500.out', res\_num\_col=1, data\_col=3, error\_col=4) \\
 \\
\# Initialise the diffusion tensor. \\
diffusion\_tensor.init(10e-9, fixed=True) \\
 \\
\# Create all attached protons. \\
sequence.attach\_protons() \\
 \\
\# Define the magnetic dipole-dipole relaxation interaction. \\
dipole\_pair.define(spin\_id1=`@N', spin\_id2=`@H', direct\_bond=True) \\
dipole\_pair.set\_dist(spin\_id1=`@N', spin\_id2=`@H', ave\_dist=1.02 * 1e-10) \\
\#dipole\_pair.unit\_vectors() \\
 \\
\# Define the CSA relaxation interaction. \\
value.set(-172 * 1e-6, `csa') \\
 \\
\# Select the model-free model. \\
model\_free.select\_model(model=name) \\
 \\
\# Grid search. \\
grid\_search(inc=11) \\
 \\
\# Minimise. \\
minimise(`newton') \\
 \\
\# Monte Carlo simulations. \\
monte\_carlo.setup(number=100) \\
monte\_carlo.create\_data() \\
monte\_carlo.initial\_values() \\
minimise(`newton') \\
eliminate() \\
monte\_carlo.error\_analysis() \\
 \\
\# Finish. \\
results.write(file=`results', force=True) \\
state.save(`save', force=True)
\end{exampleenv}


% The explanation.
%~~~~~~~~~~~~~~~~~

\subsection{Explanation}

The above script consists of three major sections:

\begin{description}
\item[Loading of data] Firstly a data pipe called \promptstring{m4} is created to hold all of the analysis data.  Then the $^{15}$N spin system data consisting of molecule, residue, and spin information is loaded into relax from the columns of the \file{noe.500.out} file, assuming that only residue numbers and names are present and are in the first and second columns respectively.  The options of this \uf{sequence.read} user function allow the molecule name, residue number, residue name, spin number, or spin name columns to be specified if desired.  The $^{15}$N spin is then set up using the \uf{spin} user functions.  The next part is to load all of the relaxation data, to set up the initial diffusion tensor, create the $^1$H spins required for the magnetic dipole-dipole interaction, and to set up the magnetic dipole-dipole and CSA relaxation mechanisms.  Finally the model-free model \promptstring{m4} is chosen.
\item[Optimisation] The optimisation of model-free models requires an initial grid search to find a position close to the minimum, followed by the high precision Newton optimisation together with the Method of Multipliers constraint algorithm \citep{dAuvergneGooley08a}.  Errors are propagated from the relaxation data to the model-free parameters via Monte Carlo simulations which is a multi-step process in relax (designed for flexibility and to teach how the simulations are constructed and carried out).
\item[Data output] The last stage consists of writing out the XML formatted results file which contains all of the data in the current data pipe, as well as the XML formatted save file which contains not only the current data pipe data but all of the relax data store data.  Both files can be loaded back into relax later on.
\end{description}



% Optimisation of all model-free models.
%%%%%%%%%%%%%%%%%%%%%%%%%%%%%%%%%%%%%%%%

\section{Optimisation of all model-free models}


% The sample script.
%~~~~~~~~~~~~~~~~~~~

\subsection{The sample script}

The sample script which demonstrates the optimisation of all model-free models from $m0$ to $m9$ of individual spins is \file{model\_free/mf\_multimodel.py}.  The important parts of the script are:

\begin{exampleenv}
\# Set the data pipe names (also the names of preset model-free models). \\
pipes = [`m0', `m1', `m2', `m3', `m4', `m5', `m6', `m7', `m8', `m9'] \\
 \\
\# Loop over the pipes. \\
for name in pipes: \\
\hspace*{4ex} \# Create the data pipe. \\
\hspace*{4ex} pipe.create(name, `mf') \\
 \\
\hspace*{4ex} \# Set up the 15N spins. \\
\hspace*{4ex} sequence.read(`noe.500.out', res\_num\_col=1) \\
\hspace*{4ex} spin.name(`N') \\
\hspace*{4ex} spin.element(element=`N', spin\_id=`@N') \\
\hspace*{4ex} spin.isotope(`15N', spin\_id=`@N') \\
 \\
\hspace*{4ex} \# Load a PDB file. \\
\hspace*{4ex} structure.read\_pdb(`example.pdb') \\
 \\
\hspace*{4ex} \# Load the relaxation data. \\
\hspace*{4ex} relax\_data.read(ri\_id=`R1\_600',  ri\_type=`R1',  frq=600.0*1e6, file=`r1.600.out', res\_num\_col=1, data\_col=3, error\_col=4) \\
\hspace*{4ex} relax\_data.read(ri\_id=`R2\_600',  ri\_type=`R2',  frq=600.0*1e6, file=`r2.600.out', res\_num\_col=1, data\_col=3, error\_col=4) \\
\hspace*{4ex} relax\_data.read(ri\_id=`NOE\_600', ri\_type=`NOE', frq=600.0*1e6, file=`noe.600.out', res\_num\_col=1, data\_col=3, error\_col=4) \\
\hspace*{4ex} relax\_data.read(ri\_id=`R1\_500',  ri\_type=`R1',  frq=500.0*1e6, file=`r1.500.out', res\_num\_col=1, data\_col=3, error\_col=4) \\
\hspace*{4ex} relax\_data.read(ri\_id=`R2\_500',  ri\_type=`R2',  frq=500.0*1e6, file=`r2.500.out', res\_num\_col=1, data\_col=3, error\_col=4) \\
\hspace*{4ex} relax\_data.read(ri\_id=`NOE\_500', ri\_type=`NOE', frq=500.0*1e6, file=`noe.500.out', res\_num\_col=1, data\_col=3, error\_col=4) \\
 \\
\hspace*{4ex} \# Set up the diffusion tensor. \\
\hspace*{4ex} diffusion\_tensor.init(1e-8, fixed=True) \\
 \\
\hspace*{4ex} \# Generate the 1H spins for the magnetic dipole-dipole relaxation interaction. \\
\hspace*{4ex} sequence.attach\_protons() \\
 \\
\hspace*{4ex} \# Define the magnetic dipole-dipole relaxation interaction. \\
\hspace*{4ex} dipole\_pair.define(spin\_id1=`@N', spin\_id2=`@H', direct\_bond=True) \\
\hspace*{4ex} dipole\_pair.set\_dist(spin\_id1=`@N', spin\_id2=`@H', ave\_dist=1.02 * 1e-10) \\
\hspace*{4ex} structure.get\_pos(`@N') \\
\hspace*{4ex} structure.get\_pos(`@H') \\
\hspace*{4ex} dipole\_pair.unit\_vectors() \\
 \\
\hspace*{4ex} \# Define the chemical shift relaxation interaction. \\
\hspace*{4ex} value.set(-172 * 1e-6, `csa', spin\_id=`@N') \\
 \\
\hspace*{4ex} \# Select the model-free model. \\
\hspace*{4ex} model\_free.select\_model(model=name) \\
 \\
\hspace*{4ex} \# Minimise. \\
\hspace*{4ex} grid\_search(inc=11) \\
\hspace*{4ex} minimise(`newton') \\
 \\
\hspace*{4ex} \# Write the results. \\
\hspace*{4ex} results.write(file=`results', force=True) \\
 \\
\# Save the program state. \\
state.save(`save', force=True)
\end{exampleenv}


% The explanation.
%~~~~~~~~~~~~~~~~~

\subsection{Explanation}

The above script is very similar in spirit to the previous single model script in section~\ref{sect: single mf model} on page~\pageref{sect: single mf model}.  The major difference is that this script loops over all of the model-free models, saving all of the results in the \file{save.bz2} file.



% Model-free model selection.
%%%%%%%%%%%%%%%%%%%%%%%%%%%%%

\section{Model-free model selection}


% The sample script.
%~~~~~~~~~~~~~~~~~~~

\subsection{The sample script}

The sample script which demonstrates both model-free model elimination and model-free model selection between models from $m0$ to $m9$ is \file{model\_free/modsel.py}.  The text of the script is:

\begin{exampleenv}
\# Set the data pipe names. \\
pipes = [`m0', `m1', `m2', `m3', `m4', `m5', `m6', `m7', `m8', `m9'] \\
 \\
\# Loop over the data pipe names. \\
for name in pipes: \\
\hspace*{4ex} print ``$\backslash$n$\backslash$n\# '' + name + `` \#'' \\
 \\
\hspace*{4ex} \# Create the data pipe. \\
\hspace*{4ex} pipe.create(name, `mf') \\
 \\
\hspace*{4ex} \# Reload precalculated results from the file `m1/results', etc. \\
\hspace*{4ex} results.read(file=`results', dir=name) \\
 \\
\# Model elimination. \\
eliminate() \\
 \\
\# Model selection. \\
model\_selection(method=`AIC', modsel\_pipe=`aic') \\
 \\
\# Write the results. \\
state.save(`save', force=True) \\
results.write(file=`results', force=True)
\end{exampleenv}


% The explanation.
%~~~~~~~~~~~~~~~~~

\subsection{Explanation}

This script is designed to be used in conjunction with the \file{model\_free/mf\_multimodel.py} script in the previous section.  It will load all of the results files from the previous script and then perform the following:

\begin{description}
\item[Model-free model elimination]  The optimisation of model-free models performed by the previous script will fail for certain data sets together with certain models.  To ensure that these models are never selected, they are removed from the analysis (see \citet{dAuvergneGooley06}).
\item[Model-free model selection]  The AIC model selection as described in \citet{dAuvergneGooley03} will be used to determine which model-free model best describes the relaxation data.
\item[Data output]  Finally both a save state and result file will be created.
\end{description}

These three sample scripts describe the basic components of model-free analysis.  However a full analysis requires the construction of a much more complex iterative procedure.  The following sections will describe both the original diffusion seeded approaches as well as the new model-free protocol built into relax.


% The methodology of Mandel et al., 1995.
%%%%%%%%%%%%%%%%%%%%%%%%%%%%%%%%%%%%%%%%%

\section{The methodology of Mandel et al., 1995}
\label{sect: Mandel 1995}

% Mandel et al., 1995 figure.
\begin{figure}
\centerline{\includegraphics[width=0.8\textwidth, bb=0 0 436 539]{images/model_free/mandel95.eps.gz}}
\caption[A schematic of the model-free optimisation protocol of Mandel et al., 1995]{A schematic of the model-free optimisation protocol of \citet{Mandel95}.  This specific protocol is for single field strength data.  The initial diffusion tensor estimate is calculated using the $\Rtwo/\Rone$ ratio.  The diffusion parameters of $\Diffset$ are held constant while model-free models $m1$ to $m5$ (\ref{model: m1}--\ref{model: m5}) of the set $\Mfset_i$ for each spin $i$ are optimised and 500 Monte Carlo simulations executed.  Using a web of ANOVA statistical tests, specifically $\chi^2$ and F-tests, a step-up hypothesis testing model selection procedure is used to choose the best model-free model.  These steps are repeated for all spins of the molecule.  The global model $\Space$, the union of $\Diffset$ and all $\Mfset_i$, is then optimised.  These steps are repeated until convergence of the global model.  The iterative process is repeated for both isotropic diffusion (sphere) and anisotropic diffusion (spheroid).} \label{fig: Mandel et al.}
\end{figure}

By presenting a systematic methodology for obtaining a consistent model-free description of the dynamics of the system, the manuscript of \citet{Mandel95} revolutionised the application of model-free analysis.  The full protocol is presented in Figure~\ref{fig: Mandel et al.}.

All of the data analysis techniques required for this protocol can be implemented within relax.  The chi-squared distributions required for the chi-squared tests are constructed by Modelfree4 from the Monte Carlo simulations.  If the optimisation algorithms and Monte Carlo simulations built into relax are utilised, then the relax script will need to construct the chi-squared distributions from the results as this is not yet coded into relax.  The specific step-up hypothesis testing model selection of \citet{Mandel95} is available through the \uf{model\_selection} user function.  Coding the rest of the protocol into a script should be straightforward.

To implement this analysis, a number of scripts would need to be written.  There is no sample script in relax for performing this analysis.  The simple sample scripts from above would need to be extended.  For example a starting script for determining the initial diffusion tensor estimates based on the R1/R2 ratio of \citet{Kay89} would have to be written.  The tensor from this script could then be feed into the \file{model\_free/mf\_multimodel.py} script, followed by the \file{model\_free/modsel.py} script, and then a third script written to optimise the diffusion tensor.  A master script could be written first run the initial diffusion tensor script, then to iteratively execute the last three scripts until convergence, and finally to select the best diffusion model (see Figure~\ref{fig: Mandel et al.}).  Alternatively, these could all be combined into one super script.



% The diffusion seeded paradigm.
%%%%%%%%%%%%%%%%%%%%%%%%%%%%%%%%

\section{The diffusion seeded paradigm}
\label{sect: diffusion seeded paradigm}

Ever since the original Lipari and Szabo papers \citep{LipariSzabo82a, LipariSzabo82b}, the question of how to obtain the model-free description of the system has followed the route in which the diffusion tensor is initially estimated.  Using this rough estimate, the model-free models are optimised for each spin system $i$, the best model selected, and then the global model $\Space$ of the diffusion model $\Diffset$ with each model-free model $\Mfset_i$ is optimised.  This procedure is then repeated using the diffusion tensor parameters of $\Space$ as the initial input.  Finally the global model is selected.  The full protocol, when combined with AIC model selection \citep{dAuvergneGooley03}, is illustrated in Figure~\ref{fig: init diff estimate}.


% The diffusion seeded paradigm figure.
\begin{figure}
\centerline{\includegraphics[width=0.8\textwidth, bb=0 0 437 523]{images/model_free/init_diff_est.eps.gz}}
\caption[Model-free analysis using the diffusion seeded paradigm]{A schematic of model-free analysis using the diffusion seeded paradigm -- the initial diffusion tensor estimate -- together with AIC model selection and model elimination.  The initial estimates of the parameters of $\Diffset$ are held constant while model-free models $m0$ to $m9$ (\ref{model: m0}--\ref{model: m9}) of the set $\Mfset_i$ for each spin system $i$ are optimised, model elimination applied to remove failed models, and AIC model selection used to determine the best model.  The global model $\Space$, the union of $\Diffset$ and all $\Mfset_i$, is then optimised.  These steps are repeated until convergence of the global model.  The entire iterative process is repeated for each of the Brownian diffusion models.  Finally AIC model selection is used to determine the best description of the dynamics of the molecule by selecting between the global models $\Space$ including the sphere, oblate spheroid, prolate spheroid, and ellipsoid.  Once the solution has been found, Monte Carlo simulations can be utilised for error analysis.} \label{fig: init diff estimate}
\end{figure}

Again this protocol is not implemented in the relax sample scripts.  This would have to be implemented in exactly the same manner as described in the previous section, but using the AIC model selection build into relax.  Constructing this set of scripts, or a single master script, would be much easier than the \citet{Mandel95} protocol as Modelfree4 would not need to be used, and the handling of F-tests and chi-squared tests is avoided.


% The new model-free optimisations protocol.
%%%%%%%%%%%%%%%%%%%%%%%%%%%%%%%%%%%%%%%%%%%%

\section{The new model-free optimisation protocol}
\label{sect: new model-free protocol}

% The model-free models.
%~~~~~~~~~~~~~~~~~~~~~~~

\subsection{The model-free models}

The study of the dynamics of a macromolecule using model-free analysis to interpret the $\Rone$ and $\Rtwo$ relaxation rates together with the steady-state heteronuclear NOE brings two distinct, yet linked physical theories into play.  The Brownian rotational diffusion of the molecule is the major contributor to relaxation.  Although having less of an influence on relaxation the internal dynamics of individual nuclei within the molecule is nevertheless significant.  The model-free description of the internal motion and the global diffusion of the entire molecule are theories which are linked due to their dependence on the same relaxation data.  The model-free models for individual spin system constructed from the original and extended model-free theories \citep{LipariSzabo82a, LipariSzabo82b, Clore90a} are assembled using parametric restrictions, the dropping of insignificant parameters, and the addition of the chemical exchange parameter $R_{ex}$.  Labelled as $m0$ to $m9$ (Models~\ref{model: m0}--\ref{model: m9} on page \pageref{model: m9}) these models are an extended list of those in \citep{Fushman97, Orekhov99b, Korzhnev01, Zhuravleva04}.



% The diffusion tensor.
%~~~~~~~~~~~~~~~~~~~~~~

\subsection{The diffusion tensor}


% The ellipsoid.
\subsubsection{The ellipsoid}
\index{diffusion!ellipsoid (asymmetric)|textbf}

The most general form of Brownian rotational diffusion of macromolecules is the diffusion of an ellipsoid, a diffusion also labelled as asymmetric or fully anisotropic.  This diffusion tensor can be fully specified by the geometric parameters $\Diff_x$, $\Diff_y$, and $\Diff_z$, the eigenvalues of the tensor, as well as three orientational parameters, the Euler angles\index{Euler angles} $\alpha$, $\beta$, and $\gamma$.  The diffusion equation for an ellipsoid was derived using the reasoning of \citet{Einstein05} in the two papers of \citet{Perrin34} and \citet{Perrin36}.  Following this, \citet{Favro60} unknowingly derived the same equations as presented in \citet{Perrin36} using a pseudo quantum mechanical approach.  Borrowing heavily from \citet{Perrin36}, \citet{Woessner62} derived the correlation function relevant for NMR relaxation of a bond vector rigidly attached to an ellipsoid.  However these equations are not fully simplified and the parameter set \{$\Diff_x$, $\Diff_y$, $\Diff_z$, $\alpha$, $\beta$, $\gamma$\}, the eigenvalues and Euler angles\index{Euler angles} defining the tensor, is not optimally constructed for minimisation.  A parameter shift to the set \{$\Diff_{iso}$, $\Diff_a$, $\Diff_r$, $\alpha$, $\beta$, $\gamma$\}, whereby the three geometric parameters are respectively the isotropic, anisotropic, and rhombic components of the diffusion tensor, drastically simplifies optimisation and is how the diffusion tensor is implemented within relax.


% The spheroid.
\subsubsection{The spheroid}
\index{diffusion!spheroid (axially symmetric)|textbf}

When two of the eigenvalues of the diffusion tensor are equal the molecule diffuses as a spheroid.  This is also called axially symmetric anisotropic diffusion and can be described by the two geometric parameters $\Diff_{iso}$ and $\Diff_a$ together with the polar angle $\theta$ and azimuthal angle $\phi$ which define the unique axis of the diffusion tensor.  Two classes of spheroid can be distinguished dependent on the relative values of the eigenvalues -- the prolate and oblate spheroids.  By using parametric constraints, both tensor types can be optimised within relax.


% The sphere.
\subsubsection{The sphere}
\index{diffusion!sphere (isotropic)|textbf}

The simplest form of diffusion occurs when all three eigenvalues are equal and the molecule diffuses as a sphere.  This isotropic rotation can be characterised by the single parameter $\Diff_{iso}$ which is related to the global correlation time by the formula $1/\tau_m = 6\Diff_{iso}$ \citep{Bloembergen48}.


% The local tm model-free models.
\begin{latexonly}
    \subsubsection{The local $\tau_m$ model-free models}
\end{latexonly}
\begin{htmlonly}
    \subsubsection{The local $tau_m$ model-free models}
\end{htmlonly}

Not only can the diffusion tensor be optimised as a global model affecting all spins of the molecule but a set of model-free models can be constructed in which each spin is assumed to diffuse independently.  In these models a single local $\tau_m$ parameter approximates the true, multiexponential description of the Brownian rotational diffusion of the molecule.  Each spin of the macromolecule is treated independently.  Another set of model-free models which include the local $\tau_m$ parameter can be created and include $tm0$ to $tm9$ (Models~\ref{model: tm0}--\ref{model: tm9} on page \pageref{model: tm0}).  These are simply models $m0$ to $m9$ with the local $\tau_m$ parameter added.  These models are an extension of the ideas introduced in \citet{Barbato92} and \citet{Schurr94} whereby the model $tm2$, the original Lipari and Szabo model-free equation with a local $\tau_m$ parameter, is optimised to avoid issues with inaccurate diffusion tensor approximations.


% Determination of the diffusion tensor from the local tm parameter.
\begin{latexonly}
    \subsubsection{Determination of the diffusion tensor from the local $\tau_m$ parameter}
\end{latexonly}
\begin{htmlonly}
    \subsubsection{Determination of the diffusion tensor from the local $tau_m$ parameter}
\end{htmlonly}

In \citet{Bruschweiler95} and further investigated in \citet{Lee97}, a methodology for determining the diffusion tensor from the local $\tau_m$ parameter together with the orientation of the XH bond represented by the unit vector $\mu_i$ was presented.  A local $\tau_m$ value was obtained for each spin $i$ by optimising model $tm2$.  The $\tau_{m,i}$ values were approximated using the quadric model
\begin{equation} \label{eq: quadric}
 (6 \tau_{m,i})^{-1} = \mu_i^T Q \mu_i,
\end{equation}

\noindent where the eigenvalues of the matrix $Q$ are defined as $Q_x = (\Diff_y + \Diff_z)/2$, $Q_y = (\Diff_x + \Diff_z)/2$, and $Q_z = (\Diff_x + \Diff_y)/2$.  The diffusion tensor is then found by linear least-squares fitting.



% The universal solution.
%~~~~~~~~~~~~~~~~~~~~~~~~

\begin{latexonly}
    \subsection{The universal solution $\Univset$}
\end{latexonly}
\begin{htmlonly}
    \subsection{The universal solution $U$}
\end{htmlonly}

The complex model-free problem, in which the motions of each spin are both mathematically and statistically dependent on the diffusion tensor and vice versa, was formulated using set theory in \citet{dAuvergneGooley07}.  This paper is important for understanding the entire concept of the new protocol in relax and for truly grasping the complexity of the model-free problem.  The solution $\widehat\Univset$ to the model-free problem was derived as an element of the universal set $\Univset$, the union of the diverse model-free parameter spaces $\Space$.  Each set $\Space$ was constructed from the union of the model-free models $\Mfset$ for all spins and the diffusion parameter set $\Diffset$.  A single parameter loss on a single spin shifts optimisation to a different space $\Space$.  Ever since the seminal work of \citet{Kay89} the model-free problem has been tackled by first finding an initial estimate of the diffusion tensor and then determining the model-free dynamics of the system (see Sections~\ref{sect: Mandel 1995} on page~\pageref{sect: Mandel 1995} and~\ref{sect: diffusion seeded paradigm} on page~\pageref{sect: diffusion seeded paradigm}).  This diffusion seeded paradigm is now highly evolved and much theory has emerged to improve this path to the solution $\widehat\Univset$.  The technique can, at times, suffer from a number of issues including the two minima problem of the spheroid diffusion tensor parameter space, the appearance of artificial chemical exchange \citep{Tjandra96}, the appearance of artificial nanosecond motions \citep{Schurr94}, and the hiding of internal nanosecond motions caused by the violation of the rigidity assumption \citep{Orekhov95b, Orekhov99b, Orekhov99a}.



% Model-free analysis in reverse.
%~~~~~~~~~~~~~~~~~~~~~~~~~~~~~~~~

\subsection{Model-free analysis in reverse}

% New model-free optimisation protocol figure.
\begin{figure}
\centerline{\includegraphics[width=0.8\textwidth, bb=0 0 461 697]{images/model_free/new_protocol.eps.gz}}
\caption[A schematic of the new model-free optimisation protocol]{A schematic of the new model-free optimisation protocol.  Initially models $tm0$ to $tm9$ (\ref{model: tm0}--\ref{model: tm9}) of the set $\Localset_i$ for each spin system $i$ are optimised, model elimination used to remove failed models, and AIC model selection used to pick the best model.  Once all the $\Localset_i$ have been determined for the system the the local $\tau_m$ parameter is removed, the model-free parameters are held fixed, and the global diffusion parameters of $\Diffset$ are optimised.  These parameters are used as input for the central part of the schematic which follows the same procedure as that of Figure~\ref{fig: init diff estimate}.  Convergence is however precisely defined as identical models $\Space$, identical $\chi^2$ values, and identical parameters $\theta$ between two iterations.  The universal solution $\widehat\Univset$, the best description of the dynamics of the molecule, is determined using AIC model selection to select between the local $\tau_m$ models for all spins, the sphere, oblate spheroid, prolate spheroid, ellipsoid, and possibly hybrid models whereby multiple diffusion tensors have been applied to different parts of the molecule.} \label{fig: new protocol}
\end{figure}


A different approach was proposed in \citet{dAuvergneGooley08b} for finding the universal solution $\widehat\Univset$ of the extremely complex, convoluted model-free optimisation and modelling problem \citep{dAuvergneGooley07}, defined as
\begin{equation} \label{eq: universal solution}
 \widehat\Univset = \hat\theta \in \left\{ \Space: \min_{\hat\theta \in \Univset} \KL(\hat\theta) \right\},
  \text{\quad s.t. }
  \hat\theta = \arg \min \left\{\chi^2(\theta): \theta \in \Space \right\}.
\end{equation}

\noindent This notation says that the minimised parameter vector within the space $\Space$ which minimises the common Kullback-Leibler discrepancy\index{discrepancy!Kullback-Leibler} $\KL$ is selected from the universal set $\Univset$ as the universal solution $\widehat\Univset$.  The discrepancy of \citet{KullbackLeibler51} is a measure of how well the model fits the data, in this case how well the global model $\Space$ of the diffusion tensor together with the model-free models of all residues fits the relaxation data.  This selection is subject to the condition that $\hat\theta$ is the argument or specific parameter vector which minimises the chi-squared\index{chi-squared} function $\chi^2(\theta)$ such that $\theta$ is an element of the space $\Space$.  Whereas the minimisation of the continuous chi-squared\index{chi-squared} function within the single space $\Space$ belongs to the mathematical field of optimisation \citep{NocedalWright99}, the selection of the universe $\Space$ which minimises the discrepancy\index{discrepancy} belongs to the statistical field of model selection \citep{Akaike73,Schwarz78,LinhartZucchini86,Zucchini00,dAuvergneGooley03}.

This new model-free optimisation protocol incorporates the ideas of the local $\tau_m$ model-free model \citep{Barbato92, Schurr94} and the optimisation of the diffusion tensor using information from these models, analogously to the linear least-squares fitting of the quadric model \citep{Bruschweiler95, Lee97}.  The protocol also follows the lead of the model-free optimisation protocol presented in \citet{Butterwick04} whereby the diffusion seeded paradigm was reversed.  Rather than starting with an initial estimation of the global diffusion tensor from the set $\Diffset$ the protocol starts with the model-free parameters from $\Mfset$.

The first step of the \citet{Butterwick04} protocol is the reduced spectral density mapping of \citet{Farrow95}.  As $R_{ex}$ has been eliminated from the analysis, three model-free models corresponding to $tm1$, $tm2$, and $tm5$ (Models~\ref{model: tm1}, \ref{model: tm2}, and \ref{model: tm5} on page~\pageref{model: tm1}) are employed.  The model-free parameters are optimised using the reduced spectral density values and the best model is selected using F-tests.  The spherical, spheroidal, and ellipsoidal diffusion tensors are obtained by linear least-squares fitting of the quadric model of Equation~\eqref{eq: quadric} using the local $\tau_m$ values \citep{Bruschweiler95, Lee97}.  The best diffusion model is selected via F-tests and refined by iterative elimination of spins systems with high chi-squared values.  This tensor is used to calculate local $\tau_m$ values for each spin system, approximating the multiexponential sum of the Brownian rotational diffusion correlation function with a single exponential, using the quadric model of Equation~\eqref{eq: quadric}.  In the final step of the protocol these $\tau_m$ values are fixed and $m1$, $m2$, and $m5$ (Models~\ref{model: m1}, \ref{model: m2}, and \ref{model: m5} on page~\pageref{model: m1}) are optimised and the best model-free model selected using F-tests.

The new model-free protocol built into relax utilises the core foundation of the \citet{Butterwick04} protocol yet its divergent implementation is designed to solve the universal equation of \citet{dAuvergneGooley07} to find $\widehat\Univset$ (Equation~\ref{eq: universal solution}).  Models $tm0$ to $tm9$ (\ref{model: tm0}--\ref{model: tm9} on page \pageref{model: tm0}) in which no global diffusion parameters exist are employed to significantly collapse the complexity of the problem.  Model-free minimisation \citep{dAuvergneGooley08a}, model elimination \citep{dAuvergneGooley06}, and then AIC\index{model selection!AIC} model selection \citep{Akaike73, dAuvergneGooley03} can be carried out in the absence of the influence of global parameters.  By removing the local $\tau_m$ parameter and holding the model-free parameter values constant these models can then be used to optimise the diffusion parameters of $\Diffset$.  Model-free optimisation, model elimination, AIC model selection, and optimisation of the global model $\Space$ is iterated until convergence.  The iterations allow for sliding between different universes $\Space$ to enable the collapse of model complexity, to refine the diffusion tensor, and to find the solution within the universal set $\Univset$.  The last step is the AIC model selection between the different diffusion models.  Because the AIC criterion approximates the Kullback-Leibler discrepancy \citep{KullbackLeibler51}, central to the universal solution of Equation~\eqref{eq: universal solution}, it was chosen for all three model selection steps over BIC model selection \citep{Schwarz78, dAuvergneGooley03, Chen04}.  The new protocol avoids the problem of under-fitting\index{under-fitting} whereby artificial motions appear, avoids the problems involved in finding the initial diffusion tensor within $\Diffset$, and avoids the problem of hidden internal nanosecond motions and the inability to slide between universes to get to $\widehat\Univset$ (see \citet{dAuvergneGooley07} for more details).  The full protocol is summarised in Figure~\ref{fig: new protocol}.



% The sample script.
%~~~~~~~~~~~~~~~~~~~

\subsection{The sample script}

The sample script for performing this new analysis is \file{model\_free/dauvergne\_protocol.py}.  The full script is given below as the docstring at the start explains the practical implementation of the protocol:

\begin{exampleenv}
"""Script for black-box model-free analysis. \\
 \\
This script is designed for those who appreciate black-boxes or those who appreciate complex code.  Importantly data at multiple magnetic field strengths is essential for this analysis.  The script will need to be heavily tailored to the molecule in question by changing the variables just below this documentation.  If you would like to change how model-free analysis is performed, the code in the class Main can be changed as needed.  For a description of object-oriented coding in python using classes, functions/methods, self, etc., see the python tutorial. \\
 \\
If you have obtained this script without the program relax, please visit http://www.nmr-relax.com. \\
 \\
 \\
References \\
========== \\
 \\
The model-free optimisation methodology herein is that of: \\
 \\
    d'Auvergne, E. J. and Gooley, P. R. (2008b). Optimisation of NMR dynamic models II. A new methodology for the dual optimisation of the model-free parameters and the Brownian rotational diffusion tensor. J. Biomol. NMR, 40(2), 121-133 \\
 \\
Other references for features of this script include model-free model selection using Akaike's Information Criterion: \\
 \\
    d'Auvergne, E. J. and Gooley, P. R. (2003). The use of model selection in the model-free analysis of protein dynamics. J. Biomol. NMR, 25(1), 25-39. \\
 \\
The elimination of failed model-free models and Monte Carlo simulations: \\
 \\
    d'Auvergne, E. J. and Gooley, P. R. (2006). Model-free model elimination: A new step in the model-free dynamic analysis of NMR relaxation data. J. Biomol. NMR, 35(2), 117-135. \\
 \\
Significant model-free optimisation improvements: \\
 \\
    d'Auvergne, E. J. and Gooley, P. R. (2008a). Optimisation of NMR dynamic models I. Minimisation algorithms and their performance within the model-free and Brownian rotational diffusion spaces. J. Biomol. NMR, 40(2), 107-109. \\
 \\
Rather than searching for the lowest chi-squared value, this script searches for the model with the lowest AIC criterion.  This complex multi-universe, multi-dimensional search is formulated using set theory as the universal solution: \\
 \\
    d'Auvergne, E. J. and Gooley, P. R. (2007). Set theory formulation of the model-free problem and the diffusion seeded model-free paradigm. 3(7), 483-494. \\
 \\
The basic three references for the original and extended model-free theories are: \\
 \\
    Lipari, G. and Szabo, A. (1982a). Model-free approach to the interpretation of nuclear magnetic-resonance relaxation in macromolecules I. Theory and range of validity. J. Am. Chem. Soc., 104(17), 4546-4559. \\
 \\
    Lipari, G. and Szabo, A. (1982b). Model-free approach to the interpretation of nuclear magnetic-resonance relaxation in macromolecules II. Analysis of experimental results. J. Am. Chem. Soc., 104(17), 4559-4570. \\
 \\
    Clore, G. M., Szabo, A., Bax, A., Kay, L. E., Driscoll, P. C., and Gronenborn, A.M. (1990). Deviations from the simple 2-parameter model-free approach to the interpretation of N-15 nuclear magnetic-relaxation of proteins. J. Am. Chem. Soc., 112(12), 4989-4991. \\
 \\
 \\
How to use this script \\
====================== \\
 \\
The value of the variable DIFF\_MODEL will determine the behaviour of this script.  The five diffusion models used in this script are: \\
 \\
    Model I   (MI)   - Local tm. \\
    Model II  (MII)  - Sphere. \\
    Model III (MIII) - Prolate spheroid. \\
    Model IV  (MIV)  - Oblate spheroid. \\
    Model V   (MV)   - Ellipsoid. \\
 \\
Model I must be optimised prior to any of the other diffusion models, while the Models II to V can be optimised in any order.  To select the various models, set the variable DIFF\_MODEL to the following strings: \\
 \\
    MI   - `local\_tm' \\
    MII  - `sphere' \\
    MIII - `prolate' \\
    MIV  - `oblate' \\
    MV   - `ellipsoid' \\
 \\
This approach has the advantage of eliminating the need for an initial estimate of a global diffusion tensor and removing all the problems associated with the initial estimate. \\
 \\
It is important that the number of parameters in a model does not exceed the number of relaxation data sets for that spin.  If this is the case, the list of models in the MF\_MODELS and LOCAL\_TM\_MODELS variables will need to be trimmed. \\
 \\
 \\
Model I - Local tm \\
~~~~~~~~~~~~~~~~~~ \\
 \\
This will optimise the diffusion model whereby all spin of the molecule have a local tm value, i.e. there is no global diffusion tensor.  This model needs to be optimised prior to optimising any of the other diffusion models.  Each spin is fitted to the multiple model-free models separately, where the parameter tm is included in each model. \\
 \\
AIC model selection is used to select the models for each spin. \\
 \\
 \\
Model II - Sphere \\
~~~~~~~~~~~~~~~~~ \\
 \\
This will optimise the isotropic diffusion model.  Multiple steps are required, an initial optimisation of the diffusion tensor, followed by a repetitive optimisation until convergence of the diffusion tensor.  Each of these steps requires this script to be rerun. For the initial optimisation, which will be placed in the directory `./sphere/init/', the following steps are used: \\
 \\
The model-free models and parameter values for each spin are set to those of diffusion model MI. \\
 \\
The local tm parameter is removed from the models. \\
 \\
The model-free parameters are fixed and a global spherical diffusion tensor is minimised. \\
 \\
 \\
For the repetitive optimisation, each minimisation is named from `round\_1' onwards.  The initial `round\_1' optimisation will extract the diffusion tensor from the results file in `./sphere/init/', and the results will be placed in the directory `./sphere/round\_1/'.  Each successive round will take the diffusion tensor from the previous round.  The following steps are used: \\
 \\
The global diffusion tensor is fixed and the multiple model-free models are fitted to each spin. \\
 \\
AIC model selection is used to select the models for each spin. \\
 \\
All model-free and diffusion parameters are allowed to vary and a global optimisation of all parameters is carried out. \\
 \\
 \\
Model III - Prolate spheroid \\
~~~~~~~~~~~~~~~~~~~~~~~~~~~~ \\
 \\
The methods used are identical to those of diffusion model MII, except that an axially symmetric diffusion tensor with Da >= 0 is used.  The base directory containing all the results is `./prolate/'. \\
 \\
 \\
Model IV - Oblate spheroid \\
~~~~~~~~~~~~~~~~~~~~~~~~~~ \\
 \\
The methods used are identical to those of diffusion model MII, except that an axially symmetric diffusion tensor with Da <= 0 is used.  The base directory containing all the results is `./oblate/'. \\
 \\
 \\
Model V - Ellipsoid \\
~~~~~~~~~~~~~~~~~~~ \\
 \\
The methods used are identical to those of diffusion model MII, except that a fully anisotropic diffusion tensor is used (also known as rhombic or asymmetric diffusion).  The base directory is `./ellipsoid/'. \\
 \\
 \\
 \\
Final run \\
~~~~~~~~~ \\
 \\
Once all the diffusion models have converged, the final run can be executed.  This is done by setting the variable DIFF\_MODEL to `final'.  This consists of two steps, diffusion tensor model selection, and Monte Carlo simulations.  Firstly AIC model selection is used to select between the diffusion tensor models.  Monte Carlo simulations are then run solely on this selected diffusion model.  Minimisation of the model is bypassed as it is assumed that the model is already fully optimised (if this is not the case the final run is not yet appropriate). \\
 \\
The final black-box model-free results will be placed in the file `final/results'. \\
""" \\
 \\
\# Python module imports. \\
from time import asctime, localtime \\
 \\
\# relax module imports. \\
from auto\_analyses.dauvergne\_protocol import dAuvergne\_protocol \\
 \\
 \\
\# Analysis variables. \\
\#\#\#\#\#\#\#\#\#\#\#\#\#\#\#\#\#\#\#\#\# \\
 \\
\# The diffusion model. \\
DIFF\_MODEL = `local\_tm' \\
 \\
\# The model-free models.  Do not change these unless absolutely necessary, the protocol is likely to fail if these are changed. \\
MF\_MODELS = [`m0', `m1', `m2', `m3', `m4', `m5', `m6', `m7', `m8', `m9'] \\
LOCAL\_TM\_MODELS = [`tm0', `tm1', `tm2', `tm3', `tm4', `tm5', `tm6', `tm7', `tm8', `tm9'] \\
 \\
\# The grid search size (the number of increments per dimension). \\
GRID\_INC = 11 \\
 \\
\# The optimisation technique. \\
MIN\_ALGOR = `newton' \\
 \\
\# The number of Monte Carlo simulations to be used for error analysis at the end of the analysis. \\
MC\_NUM = 500 \\
 \\
\# Automatic looping over all rounds until convergence (must be a boolean value of True or False). \\
CONV\_LOOP = True \\
 \\
 \\
 \\
\# Set up the data pipe. \\
\#\#\#\#\#\#\#\#\#\#\#\#\#\#\#\#\#\#\#\#\#\#\# \\
 \\
\# The following sequence of user function calls can be changed as needed. \\
 \\
\# Create the data pipe. \\
name = ``mf (\%s)'' \% asctime(localtime()) \\
pipe.create(name, `mf') \\
 \\
\# Load the PDB file. \\
structure.read\_pdb(`1f3y.pdb') \\
 \\
\# Set up the 15N and 1H spins. \\
structure.load\_spins(`@N', ave\_pos=True) \\
structure.load\_spins(`@H', ave\_pos=True) \\
spin.isotope(`15N', spin\_id=`@N') \\
spin.isotope(`1H', spin\_id=`@H') \\
 \\
\# Set up the 15N spins (alternative to the structure-based approach). \\
\#sequence.read(file=`noe.500.out', dir=None, mol\_name\_col=None, res\_num\_col=1, res\_name\_col=2, spin\_num\_col=None, spin\_name\_col=None) \\
\#spin.name(`N') \\
\#spin.element(element=`N', spin\_id=`@N') \\
\#spin.isotope(`15N', spin\_id=`@N') \\
 \\
\# Generate the 1H spins for the magnetic dipole-dipole relaxation interaction (alternative to the structure-based approach). \\
\#sequence.attach\_protons() \\
 \\
\# Load the relaxation data. \\
relax\_data.read(ri\_id=`R1\_600',  ri\_type=`R1',  frq=599.719*1e6, file=`r1.600.out',  mol\_name\_col=None, res\_num\_col=1, res\_name\_col=2, spin\_num\_col=None, spin\_name\_col=None, data\_col=3, error\_col=4) \\
relax\_data.read(ri\_id=`R2\_600',  ri\_type=`R2',  frq=599.719*1e6, file=`r2.600.out',  mol\_name\_col=None, res\_num\_col=1, res\_name\_col=2, spin\_num\_col=None, spin\_name\_col=None, data\_col=3, error\_col=4) \\
relax\_data.read(ri\_id=`NOE\_600', ri\_type=`NOE', frq=599.719*1e6, file=`noe.600.out', mol\_name\_col=None, res\_num\_col=1, res\_name\_col=2, spin\_num\_col=None, spin\_name\_col=None, data\_col=3, error\_col=4) \\
relax\_data.read(ri\_id=`R1\_500',  ri\_type=`R1',  frq=500.208*1e6, file=`r1.500.out',  mol\_name\_col=None, res\_num\_col=1, res\_name\_col=2, spin\_num\_col=None, spin\_name\_col=None, data\_col=3, error\_col=4) \\
relax\_data.read(ri\_id=`R2\_500',  ri\_type=`R2',  frq=500.208*1e6, file=`r2.500.out',  mol\_name\_col=None, res\_num\_col=1, res\_name\_col=2, spin\_num\_col=None, spin\_name\_col=None, data\_col=3, error\_col=4) \\
relax\_data.read(ri\_id=`NOE\_500', ri\_type=`NOE', frq=500.208*1e6, file=`noe.500.out', mol\_name\_col=None, res\_num\_col=1, res\_name\_col=2, spin\_num\_col=None, spin\_name\_col=None, data\_col=3, error\_col=4) \\
 \\
\# Deselect spins to be excluded (including unresolved and specifically excluded spins). \\
deselect.read(file=`unresolved', dir=None, spin\_id\_col=None, mol\_name\_col=None, res\_num\_col=1, res\_name\_col=None, spin\_num\_col=None, spin\_name\_col=None, sep=None, spin\_id=None, boolean=`AND', change\_all=False) \\
deselect.read(file=`exclude', spin\_id\_col=1) \\
 \\
\# Define the magnetic dipole-dipole relaxation interaction. \\
dipole\_pair.define(spin\_id1=`@N', spin\_id2=`@H', direct\_bond=True) \\
dipole\_pair.set\_dist(spin\_id1=`@N', spin\_id2=`@H', ave\_dist=1.02 * 1e-10) \\
dipole\_pair.unit\_vectors() \\
 \\
\# Define the chemical shift relaxation interaction. \\
value.set(-172 * 1e-6, `csa', spin\_id=`@N') \\
 \\
 \\
 \\
\# Execution. \\
\#\#\#\#\#\#\#\#\#\#\#\# \\
 \\
\# Do not change! \\
dAuvergne\_protocol(pipe\_name=name, diff\_model=DIFF\_MODEL, mf\_models=MF\_MODELS, local\_tm\_models=LOCAL\_TM\_MODELS, grid\_inc=GRID\_INC, min\_algor=MIN\_ALGOR, mc\_sim\_num=MC\_NUM, conv\_loop=CONV\_LOOP) \\
\end{exampleenv}


% The explanation.
%~~~~~~~~~~~~~~~~~

\subsection{Explanation}

The initialisation of data in this script is almost identical to that of the sample script for a single model-free model in section~\ref{sect: single mf model} on page~\pageref{sect: single mf model}.  Once the data is set up, then the data pipe as well as a number of user defined variables are passed into the \module{dAuvergne\_protocol} class.  This script needs to be executed multiple times for each of the diffusion models.

For a full analysis of a protein system, the analysis may require between one to two weeks to complete.  The analysis is performed as described in the previous sections and summarised in Figure~\ref{fig: new protocol}.  If you are curious, the implementation is within a very large relax script found at \file{auto\_analyses/dauvergne\_protocol.py} (which must never be changed).  This auto-analyses script hides all of the complexity of the analysis from the sample script.



% GUI.
%%%%%%

\section{The GUI auto-analysis}

From the analysis wizard (Figure~\ref{fig: screenshot: analysis wizard} on page~\pageref{fig: screenshot: analysis wizard}), the automated model-free analysis can be selected.  This analysis will use the new model-free protocol described in Section~\ref{sect: new model-free protocol} on page~\pageref{sect: new model-free protocol}.  Once the analysis is initialised, the screen should look like Figure~\ref{fig: screenshot: model-free analysis}.  The \guibutton{About} button in the bottom left will bring up a window with the same description as given in the sample script.  After reading this chapter, the use of this GUI analysis should be self explanatory (if not, then please consider filing a bug report at \href{https://gna.org/bugs/?func=additem\&group=relax}{https://gna.org/bugs/?func=additem\&group=relax} or a support request at \href{https://gna.org/support/?func=additem\&group=relax}{https://gna.org/support/?func=additem\&group=relax}).  The GUI is designed to be robust -- you should be able to set up all the input data and parameters in any order, with relax giving warning is something is missing.  The analysis will only execute once everything is correctly set up.  If this is not the case, rather than starting the analysis, clicking on the \guibutton{Execute relax} button will warn about the incorrect set up, describing what the problem is.

If the \gui{Protocol mode} field is left to the \gui{Fully automated} setting then, after clicking on \guibutton{Execute relax}, the calculation can be left for one to two weeks to complete.  It is highly recommended to check the log messages in the relax controller window, at least at the start of the analysis, to make sure that all the data is being read correctly and everything is set up as desired.  All warnings should be carefully checked as these can indicate a fatal problem.  If you would like to log all the messages into a file, relax can be run with:

\example{\$ relax -g --log my.log}

Note that the size of this log file could end up being in the gigabyte range for a model-free analysis.  To speed up the calculations, if you have access to multiple cores and/or hyper-threading, the GUI can be run using Gary Thompson's multi-processor framework.  For example on a dual-CPU with dual-core system, four calculations can be run simultaneously.  In this case, the GUI can be launched with:

\example{\$ mpirun -np 5 /usr/local/bin/relax --multi=`mpi4py' --gui}

This assumes that OpenMPI and the Python mpi4py module have been installed on your system.  If this is successful, you should only see a single relax GUI window (and not five windows) and in the relax controller, you should see text similar to:

\example{Processor fabric:  MPI 2.1 running via mpi4py with 4 slave processors \& 1 master.  Using Open MPI 1.4.3.}

If you are using a different MPI implementation, please see the documentation of that implementation to see how to launch a program in MPI mode.

Upon completion of the analysis, the save and results files for the final result will be located in the \directory{final} directory within the selected results directory.  The results files will consist of text files for each of the spin specific model-free parameters, 2D Grace plots of the model-free parameters, PyMOL and MOLMOL macros for superimposing the model-free parameter values onto the 3D structure of the molecule, and a PDB representation of the final diffusion tensor.  Further visualisations of the results are possible via the \guimenuitemone{User functions} menu entry.  For example to generate a 2D plot of order parameters for one of the other diffusion tensor results, the pipe editor window can be used to switch data pipes to the other diffusion models and then the \guimenuitemthree{User functions}{grace}{write} menu item can be selected to create the plot.
