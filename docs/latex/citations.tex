% Citations.
%%%%%%%%%%%%

\chapter{Preface - citing relax} \label{ch: citations}

The relax project is a large collection of work created by diverse authors.
It is a community driven project created by NMR spectroscopists which supports a broad range of dynamics analyses.
Care must be taken to properly cite the parts of relax that you use so that the correct authors receive the citations and credit they deserve.
The following is a breakdown of all of the citations relating to relax, including the basic citations for the various analysis types.



% The software relax.
%%%%%%%%%%%%%%%%%%%%%

\section*{The software relax}


% relax.
%~~~~~~~

\subsection*{relax references}

The primary citations for relax are:

\begin{itemize}
\item \bibentry{dAuvergneGooley08a}
\item \bibentry{dAuvergneGooley08b}
\end{itemize}


% Graphical user interface.
%~~~~~~~~~~~~~~~~~~~~~~~~~~

\subsection*{Graphical user interface reference}

The primary citation for the GUI is:

\begin{itemize}
\item \bibentry{Bieri11}
\end{itemize}


% The multi-processor.
%~~~~~~~~~~~~~~~~~~~~~

\subsection*{The multi-processor reference}

Although not published, if the multi-processor framework is used to run relax on multi-core systems, grids, or clusters, then please acknowledge the author of that code -- Gary Thompson.



% Specific analyses.
%%%%%%%%%%%%%%%%%%%%

\section*{Specific analyses}

The following subsections list the citations for the individual analysis specific parts of relax.


% Model-free analysis.
%~~~~~~~~~~~~~~~~~~~~~

\subsection*{Model-free analysis references}

The base citations for model-free theory are:

\begin{itemize}
\item \bibentry{LipariSzabo82a}
\item \bibentry{LipariSzabo82b}
\item \bibentry{Clore90a}
\end{itemize}

If the automated analysis of the \file{dauvergne\_protocol.py} sample script or the GUI model-free analysis which uses the same protocol has been used, then the following citations are all implicit:

\begin{itemize}
\item \bibentry{dAuvergneGooley03}
\item \bibentry{dAuvergneGooley06}
\item \bibentry{dAuvergneGooley07}
\item \bibentry{dAuvergneGooley08a}
\item \bibentry{dAuvergneGooley08b}
\end{itemize}

Otherwise, if model-free analysis is used in relax but not via the inbuilt automated protocol, the first reference is for model selection, the second is for eliminating failed model-free models, and the forth is for the optimisation improvements (the third and fifth are for the automated protocol).  All of the model-free implementation details of relax are covered by the PhD thesis (available as a PDF or as a printed version on Amazon.com) of:

\begin{itemize}
\item \bibentry{dAuvergne06}
\end{itemize}

The reference for the hybridisation of different global diffusion models to analyse the residual inter-domain dynamics -- a not very well documented feature of relax -- is:

\begin{itemize}
\item \bibentry{Horne07}
\end{itemize}

If the software Modelfree4 or Dasha are used as replacement optimisation engines from within relax, the citations for Modelfree4 are:

\begin{itemize}
\item \bibentry{Palmer91}
\item \bibentry{Mandel95}
\end{itemize}

and for Dasha:

\begin{itemize}
\item \bibentry{Orekhov95a}
\end{itemize}


% Consistency testing analysis.
%~~~~~~~~~~~~~~~~~~~~~~~~~~~~~~

\subsection*{Consistency testing analysis references}

The base citations for the consistency testing of NMR relaxation is:

\begin{itemize}
\item \bibentry{Fushman99}
\item \bibentry{Farrow95}
\item \bibentry{Fushman98}
\end{itemize}

The first is the main citation, whereas the next are the individual tests.  The citation for the consistency testing of NMR relaxation as implemented in relax is:

\begin{itemize}
\item \bibentry{MorinGagne09a}
\end{itemize}


% N-state model analysis.
%~~~~~~~~~~~~~~~~~~~~~~~~

\subsection*{N-state model analysis references}

Some citations demonstrating as well as presenting the use of the N-state model for diverse analyses types are:

\begin{itemize}
\item \bibentry{Sun11}
\item \bibentry{Erdelyi11}
\end{itemize}


% Reduced spectral density mapping.
%~~~~~~~~~~~~~~~~~~~~~~~~~~~~~~~~~~

\subsection*{Reduced spectral density mapping references}

The base citations for reduced spectral density mapping are:

\begin{itemize}
\item \bibentry{Farrow95}
\item \bibentry{Lefevre96}
\end{itemize}



% Generic parts of relax.
%%%%%%%%%%%%%%%%%%%%%%%%%

\section*{Generic parts of relax}

The following subsections will list the citations for the parts of relax independent of the specific analyses.


% Model selection.
%~~~~~~~~~~~~~~~~~

\subsection*{Model selection references}

The citation for the model selection component of relax is:

\begin{itemize}
\item \bibentry{dAuvergneGooley03}
\end{itemize}

The base citations for the specific model selection techniques of AIC, AICc, and BIC are respectively:

\begin{itemize}
\item \bibentry{Akaike73}
\item \bibentry{HurvichTsai89}
\item \bibentry{Schwarz78}
\end{itemize}



% Other citations.
%%%%%%%%%%%%%%%%%%

\section*{Other citations}

If you believe that other citations should be included in this chapter, please contact the relax users mailing list (relax-users at gna.org\index{mailing list!relax-users}).
